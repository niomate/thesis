\newcommand{\TODO}[1]{\ \\ \textbf{\textcolor{red}{!! #1 !!}}\\}
\chapter{Theoretical background}\label{ch:Theory}
%%%%%%%%%%%%%%%%%%%%%%%%%%%%%%%%%%%%%%%%%%  Basics %%%%%%%%%%%%%%%%%%%%%%%%%%%%%%%%%%%%%%%%%%%%
\section{Basics}\label{sec:Basics}
The concepts used in this thesis require some prior knowledge about basic calculus and linear
algebra as well as some more advanced topics that will be introduced in the following sections.
But before introducing corner detection and diffusion, we have to first define what an image is
mathematically.\\
A \textit{grey value image} is defined as a function $f: \Omega \rightarrow \mathbb{R}$ where
$\Omega \subset \mathbb{R}^2$ is a rectangular subset of $\mathbb{R}^2$ of size $n_x\times n_y$,
wheras a
\textit{colour image} is defined as a vector-valued function $f: \Omega \rightarrow \mathbb{R}^3$.
For the sake of simplicity, we will focus on grey value images as most of the results can easily be
transferred to vector-valued images.\\
\textbf{Notation:} Instead of writing $(x, y)$, I will use $\boldsymbol x := (x, y)$ most of the
time, as it makes most equations and definitions more readable. Furthermore, lowercase bold letters will denote vectors and uppercase bold letters will
denote matrices.
\subsection{Image gradient}
One of the most important operations on functions in image processing is \textit{partial
    differentiation}.
The partial derivative of an image $f: \Omega \rightarrow [0, 255]$ in $x$-direction is herein denoted as $f_x$ or
synonymously as $\partial_x f$ and defined as
\begin{equation}
    f_x(x, y) = \partial_x f (x, y) = \frac{\partial f}{\partial x} (x, y) := \lim_{h \to 0}\frac{f(x+h, y) -f(x, y)}{h} 
\end{equation}
The \textit{gradient} of an image $f$ is the vector containing both partial image derivatives.
In multivariable calculus, the gradient of a function is an important tool to find the (both local
and global) extrema of a function similar to the first derivative for a function with a single
variable.
\begin{equation}
    \textbf{grad}(f) = \boldsymbol\nabla f := \left(f_x, f_y\right)^\top
\end{equation}
The gradient always points in the direction of the steepest ascent/descent, it is the tangent
vector to the surface at the given location\cite{mfi3}.
Note that the gradient of a function is a vector-valued function and not a vector.

\subsection{Convolution}
Another operation from calculus that we will need is the \textit{convolution operator}.\\
\begin{equation}
    (f * g)(\boldsymbol x) := \int\limits_{\mathbb{R}^2} f(\boldsymbol x-\boldsymbol y)g(\boldsymbol y)d\boldsymbol y\label{eq:2DConv}
\end{equation}
Convolution is especially useful in image and signal processing to design so called linear filters
such as a moving average or smoothing operation\cite{ipcv19-02,dic18-02}. As a matter of fact, in a later
section we will need the convolution as a tool to smooth our image to reduce noise artifacts. To
achieve this, we will use a \textit{Gaussian convolution}, i.e.\ a convolution with a
\textit{Gaussian kernel} which is basically just a two-dimensional Gaussian function with a certain
standard deviation\cite{ipcv19-02}:
\begin{equation}
    K_\sigma (\boldsymbol x) := \frac{1}{2\pi\sigma^2}\exp\left(\frac{-\lVert\boldsymbol
            x\rVert_2^2}{2\sigma^2}\right)
\end{equation}
where $\lVert \cdot \rVert_2$ denotes the \textit{Euclidean norm}.
For the rest of this thesis, an image $f$ convolved with a Gaussian with standard deviation $\sigma$
will be denoted by \[f_\sigma := K_\sigma * f\]
Note that because of the symmetry of the
convolution, it would have been perfectly fine to write it as $f * K_\sigma$.

%%%%%%%%%%%%%%%%%%%%%%%%%%%%%%%%%%%%%%  Structure Tensor %%%%%%%%%%%%%%%%%%%%%%%%%%%%%%%%%%%%%%
\section{The Structure Tensor}\label{sec:Structure}
For some applications only the gradient of an image does not give us enough information. The
gradient on its own is mostly just used as an edge detector, hence we need to come up with
something else for e.g.\ corner detection\cite{ipcv19-13}. One option is the so called \textit{structure tensor}, a
matrix that contains information about the surrounding region at a specific position. With the
structure tensor, or rather its eigenvalues (cf.~\ref{sub:Corner}), one is able to
distinguish between flat regions, edges and corners.

\subsection{Definition}
\TODO{Reconsider the definition, maybe rewrite this paragraph later}
The structure tensor is defined as a matrix whose eigenvectors tell us the direction of
both the largest and smallest grey value change. Mathematically, we can model this
as an optimisation problem:\\
Let $u$ be a grey value image.
We want to find a unit vector $\mathbf{n} \in \mathbb{R}^2$ that is `most parallel' or `most orthogonal' to the
gradient $\boldsymbol\nabla u$ within a circle of radius $\rho > 0$, i.e. one wants to optimise the
function
\begin{align}
    E(\mathbf{n}) &= \int\limits_{B_\rho(\boldsymbol x)} \left(\mathbf{n}^\top\boldsymbol\nabla
        u\right)^2d\boldsymbol x'\\
    &= \mathbf{n}^\top \left(\int\limits_{B_\rho(\boldsymbol x)} \boldsymbol\nabla u \boldsymbol\nabla
        u^\top d\boldsymbol x' \right) \mathbf{n}\label{eq:QuadForm}
\end{align}
This function is also called the \textit{local autocorrelation function/local average
    contrast}\cite{harris88, ipcv19-13}.
Since~\eqref{eq:QuadForm} is a quadratic form of the matrix
\[M_\rho(\boldsymbol\nabla u) := \int\limits_{B_\rho(\boldsymbol x)} \boldsymbol\nabla u \boldsymbol\nabla
    u^\top d\boldsymbol x'\]
such an optimal unit vector is by definition also the eigenvector to the smallest/largest
eigenvalue of $M_\rho(\boldsymbol\nabla u)$\cite{ipcv19-13}.
The matrix $M_\rho(\boldsymbol\nabla u)$ can also be seen as a component-wise convolution with the indicator
function
\[b_\rho(\boldsymbol x) = \begin{cases} 1 & \lVert \boldsymbol x\rVert_2^2 \leq \rho^2\\ 0 & \text{else} \end{cases}\]
However, as the author stated in \cite{harris88}, using this \textit{binary window function} leads
to a noisy response and they therefore suggest using a \textit{Gaussian window function} with standard
deviation $\rho$. This parameter is also
called the \textit{integration scale} and determines how localised the structure information
is\cite{ipcv19-13}.
This ultimately leads to the definition
\begin{equation}
    \mathbf{J}_\rho(\boldsymbol\nabla u) := K_\rho * (\boldsymbol\nabla u\boldsymbol\nabla u^\top)
\end{equation}
It is important to state that almost always, one uses a smoothed or \textit{regularised} image instead of the
original unregularised form in order to reduce numeric instabilities caused by
differentiation\cite{ipcv19-12}. The definition then becomes
\begin{equation}
    \mathbf{J}_\rho(\boldsymbol\nabla u_\sigma) := K_\rho * (\boldsymbol\nabla u_\sigma\boldsymbol\nabla
    u_\sigma^\top)\label{def:StructTensor}
\end{equation}
To keep things simpler, I will omit the brackets and just simply use $\mathbf{J}_\rho$ as the
structure tensor.\\
\subsection{Usage in Corner Detection}\label{sub:Corner}
The structure tensor is a symmetric matrix and thus possesses orthonormal eigenvectors $\boldsymbol v_1,
\boldsymbol v_2$ with real-valued eigenvalues $\lambda_1, \lambda_2 \geq 0$. \cite{ipcv19-13} As
mentioned in the preface to this section, we can use these eigenvalues to distinguish between
corners, edges and flat regions as seen in figure \ref{fig:Structure}.
In total, we have to deal with 3 different cases:
\begin{enumerate}
    \item $\lambda_1, \lambda_2$ are both small $\rightarrow$ flat region
    \item one of the eigenvalues is significantly larger than the other one $\rightarrow$ edge
    \item both eigenvalues are significantly larger than 0 $\rightarrow$ corner
\end{enumerate}
If one looks at the eigenvalues as indicators of how much the grey value shifts in the
corresponding
direction, then the classification makes perfect sense. If both eigenvalues are small, then the
grey value does not shift much in either direction, thus the area does not contain any features.
In the case that one is much larger than the other one, there is an edge in direction of the
eigenvector of the larger value since the largest grey value shift is in exactly this direction.
For the last case it should be obvious why this refers to a corner region. When both eigenvalues
are large, there is a large grey value shift in either direction, therefore there has to be a
corner.\\
\begin{figure}[h]
    \centering
    \includegraphics[width=0.6\linewidth]{../Images/structure_tensor.png}
    \caption{Visualisation of distinction of image features using the eigenvalues of the structure
        tensor. $\alpha, \beta$ are equivalent to the eigenvalues $\lambda_1, \lambda_2$. Source: \cite{harris88}}\label{fig:Structure}
\end{figure}
There are several approaches to find out which case applies at the current position. The biggest
challenge here is to differentiate between edges and corners, i.e. we have to find out whether both
eigenvalues are meaningfully larger than 0 and if one is larger than the other.\\
The most intuitive approach is the one by Tomasi and Kanade, sometimes also called Shi-Tomasi
corner detector. It simply compares the smaller eigenvalue against some artificial
threshold. The set of local maxima is then the set of corners for the image\cite{shitomasi94}.
However, this approach requires to compute both eigenvalues and can thus be fairly expensive.\\
\TODO{Find sources for Rohr and Förstner}
A cheaper approach would be to either threshold the trace \[tr(\mathbf{J}_\rho) := j_{1, 1} + j_{2,
        2} = \lambda_1 + \lambda_2\] as proposed by Rohr, 1987 or the determinant \[det(\mathbf{J}_\rho) := j_{1, 1}j_{2, 2} -
    j_{1, 2}^2 = \lambda_1\lambda_2\] as proposed by Harris and F\"orstner, 1988 and 1986
respectively\cite{harris88}. Both of these approaches do not need to explicitly compute the eigenvalues of
the structure tensor and are thus not as computationally invested. Another difference between both 
approaches is that the first one requires the trace by itself to be a local maximum whereas in the
second approach, $(det(\mathbf{J}_\rho))/(tr(\mathbf{J}_\rho))$ needs to be a local maximum.\\

For the detection of relevant corners in the data selection phase, I mainly used the approach of F\"orstner/Harris as well as
the approach of Rohr even though the Tomasi-Kanade approach was an option and has also been
tested as we will see later in chapter \ref{ch:Results}. However, it has not proven as successful
as the other two methods during the initial testing phase.
%%%%%%%%%%%%%%%%%%%%%%%%%%%%%%%%%%%%%  Diffusion %%%%%%%%%%%%%%%%%%%%%%%%%%%%%%%%%%%%%%%%%%%%%%
\section{Diffusion}\label{sec:Diffusion}
The concept of diffusion is omnipresent in the physical world. It describes, in the broadest sense
possible, how particles distribute in a certain medium. This could be anything from heat in air to
ink in water. But this is not its only use. It is applicable in many more fields ranging from
natural sciences to finance and economics. In this chapter, we will see how diffusion applies to
image processing and what benefits we gain from it. Furthermore, I will go over some basic ideas to
introduce diffusion mathematically and subsequently explain different types of diffusion commonly
found and used in image processing.
\subsection{A short note on scale spaces}
Before we begin to talk about diffusion, I will use this opportunity to shortly introduce scale
spaces and explain how they are useful to diffusion processes in image processing and image
processing in general.\\
A scale space is generally defined as a family of images with a time parameter $t$ that become
increasingly `simpler' as $t \to \infty$. The point of a structure like this is, that certain image
features do only exist at a specific scale or a range of scales and that it is therefore beneficial
to the understanding of an image to basically have a hierarchy of features.\\
Mathematically, a scale
space needs to meet certain requirements like the semi-group property, maximum-minimum principles
and others.
A full explanation of all the scale space requirements would be beyond the scope of this thesis
and is frankly not needed to understand the following sections.\\
To embed an image $f:\mathbb{R}^2 \rightarrow \mathbb{R}$ in a scale space, we need a smoothing
operator $T$ that is applied iteratively to the image. One such an operation is the Gaussian
convolution: 
\begin{equation}
    T_tf := K_{\sqrt{2t}} * f
\end{equation}
This operation spans the \textit{Gaussian scale space}, one of if not the the oldest and best
understood scale space. In the western world, it was first mentioned by Witkin et
al.\cite{witkin84} 1984, but as was later found out, researchers in Japan already proposed it in
1963\cite{weickert-ishikawa}.\\
Scale spaces as a tool are very important to diffusion filtering, since diffusion is an iterative
process and as such benefits from the notion of a scale spaces. Embedding an image in such a scale
space allows us to develop a theory of evolving the image over time as we will see in the next
sections, where I will first show the basic idea behind the general diffusion equation and
subsequently introduce different types of diffusion important to image processing.
To avoid confusion we define the gradient for scale spaces as the \textit{spatial gradient} 
\begin{equation}
    \boldsymbol\nabla u(x, y, t) = \begin{pmatrix}
        \partial_x u(x, y, t)\\
        \partial_y u(x, y, t)
    \end{pmatrix} 
\end{equation}
instead of the spatiotemporal gradient.

\subsection{Mathematical background}
To get a glimpse of the basic idea of diffusion, we have to take a small dive into the world of
physics.
As mentioned in the introduction to this section, diffusion is used to describe processes of
particle or concentration distribution. The differences in concentration are evened out by a flow
or \textit{flux} that is aimed from high concentration areas to areas with low concentration. This
principle is stated by \textit{Fick's law}\eqref{eq:Fick} (hence this type of diffusion is called \textit{Fickian
diffusion}):
\begin{equation}
    \boldsymbol j = -\boldsymbol D\cdot\boldsymbol\nabla u\label{eq:Fick}
\end{equation}
Here, the flow from high to low concentration areas is modelled by a flow whose direction is
proportional to the inverse direction of the largest change in concentration, i.e. the gradient.
While most of the time, the gradient determines the direction of the flux, the so called
\textit{diffusion tensor} $\boldsymbol D$ determines its strength. In general, this diffusion
tensor is defined as a
$2\times2$ positive definite matrix, but as we will see later, it can be reduced to a scalar 
valued function in more simple cases.
In more complex cases, also called \textit{anisotropic}, the diffusion tensor can also adjust the
direction of the flow. We will see how this works in the section about \textit{nonlinear
anisotropic diffusion}.\\
Another important principle for diffusion is the conservation of mass, since mass cannot be created
out of thin air and similarly cannot simply vanish.
This principle is given by the equation
\begin{equation}
    \partial_t u = - \textnormal{div}(\boldsymbol j)\label{eq:Conservation}
\end{equation}
where $u: \Omega \times [0, \infty) \rightarrow \mathbb{R}$ is a function of time and space and div
is the \textit{divergence operator}
\[\textnormal{div}(\boldsymbol j) := \partial_x j_1 + \partial_y j_2\]
\begin{figure}[h]
    \centering
    \includegraphics[width=0.5\linewidth]{../Images/divergence2.png}
    \caption{Visualisation of the divergence operator. Original image \cite{img-divergence}
    edited with GIMP}\label{fig:Divergence}
\end{figure}
In simple terms, the divergence of the flux measures whether the concentration at the current
location diverges, i.e. diffuses away from the current location, or converges, i.e. moves in
towards the current location.
Together, equations~\eqref{eq:Fick} and \eqref{eq:Conservation} then form the general \textit{diffusion
    equation}\cite{dic18-02, weickert96}
\begin{equation}
    \partial_t u = \textnormal{div}(\boldsymbol D\cdot\boldsymbol\nabla u)\label{eq:Diffusion}
\end{equation}
%To solve this equation, it is embedded in the initial value problem
%\begin{align}
    %&\partial_t u = \textnormal{div}(\boldsymbol D\boldsymbol\nabla u)&\textnormal{on}\ 
    %\Omega\times[0, \infty)\\
    %&u(\boldsymbol x, 0) = f(\boldsymbol x)&\textnormal{on}\ \Omega\\
    %&\boldsymbol n^\top \boldsymbol D\boldsymbol\nabla u = 0&\textnormal{on}\ 
    %\partial\Omega\times[0, \infty)
%\end{align}
The solution to this equation can be interpreted as an image embedded in a scale space with the
evolution time as the scale parameter. To get different scale spaces ergo different diffusion
processes, one can alter the diffusion tensor in certain ways.
In the following, we will see how different diffusion processes are characterized and what they are
generally most useful for in image processing.

\subsubsection*{Linear isotropic diffusion}
In this first and simplest case, the diffusion tensor is replaced by a constant diffusivity
resulting in equal smoothing all across the image. The constant diffusivity can also be expressed
as the identity matrix $\boldsymbol I$ which simplifies the diffusion equation to the 2D\textit{
heat equation}
\begin{equation}
    \partial_t u = \Delta u = \textnormal{div}(\boldsymbol\nabla u)
\end{equation}
which can be solved analytically. The solution to this equation has proven to be equivalent to
a Gaussian convolution, thus this type of diffusion also produces a Gaussian scale space.\\
Obviously, homogeneous smoothing is not very useful if one wants to create a filter that is able to
enhance edges. Still, it is the best understood and oldest scale space and used for many
applications in image processing. One example is optical character recognition.
However, to achieve edge enhancing properties, we need to look into nonlinear methods.

\subsubsection*{Nonlinear isotropic diffusion}
Previously, we replaced the diffusion tensor by the identity matrix, which resulted in the same
amount of smoothing at every location.
To get spatially varying smoothing, we have to introduce a \textit{diffusivity} function that
depends continuously on the input image. 
\begin{align}
    \boldsymbol D &= g(\lVert\boldsymbol\nabla u\rVert_2^2)\boldsymbol I\\
    \Rightarrow \partial_t u & = \textnormal{div}(g(\lVert\boldsymbol\nabla u\rVert_2^2)\boldsymbol\nabla u)
\end{align}
The diffusivity function determines how much the image should be smoothed at the current location.
The most well known choice for a diffusivity function was proposed by Perona and Malik 
\cite{perona-malik}
\begin{equation}
    g(s^2) = \frac{1}{1 + s^2/\lambda^2}\label{def:Diffusivity}
\end{equation}
The interesting part about this diffusivity function is that it distinguishes between edges and
non-edges or flat areas according to the \textit{contrast parameter} $\lambda$ using the gradient
magnitude $\lVert\boldsymbol\nabla u\rVert_2^2$ as a \textit{fuzzy edge detector}\cite{dic18-04}. An important part
to understanding why this leads to an edge enhancing effect is the flux function
arising from this specific diffusivity. In general, the flux function is simply defined as 
\begin{equation}
    \Phi(s) = sg(s^2)
\end{equation}
which in this particular case comes down to
\begin{equation}
    \Phi(s) =\frac{s}{1 + s^2/\lambda^2}\label{def:Flux}
\end{equation}
\begin{figure}
    \includegraphics[width=\linewidth]{../Images/diffflux.png}
    \caption{\textbf{Left:} Diffusivity mentioned in~\eqref{def:Diffusivity}. \textbf{Right:} Flux
    function\eqref{def:Flux}. Source: \cite{dic18-04}}\label{fig:DiffFlux}
\end{figure}
As we already know, the flux describes the change in concentration at each position. The edge
enhancing effect comes from the so called \textit{backwards diffusion} which happens when the
gradient magnitude surpasses the contrast parameter as seen in \ref{fig:DiffFlux}. Backwards
diffusion is basically the opposite from `normal' diffusion in a sense that it sharpens image
features instead of smoothing them.
If we look at the derivative of the flux function this will become more obvious:
\begin{equation}
    \Phi'(s) = \frac{d}{ds} \left(\frac{s}{1 + s^2/\lambda^2}\right) = 
    \frac{1 - s^2/\lambda^2}{\left(1 + s^2/\lambda^2\right)^2}
\end{equation}
As we see, we have a maximum in the flux function at $s = \lambda$. Furthermore we can extract from
the above equation that $\Phi'(s) < 0$ for $s < \lambda$ and $\Phi'(s) > 0$ for $s > \lambda$. This
can also be seen in \ref{fig:DiffFlux}. This distinction between an increasing and decreasing flux
function for values left and right from the contrast parameter causes the previously mentioned
edge enhancing effect.
However, this particular process is not \textit{well-posed}\cite{weickert96}, i.e. it is very
sensitive to high frequent noise. To solve this, it was proposed to use Gaussian smoothing for
\textit{regularisation}, i.e. to convolve the original image with a Gaussian kernel to damp the high
frequencies that cause numerical instabilities\cite{catte-lions-morel92}.

\subsubsection{Nonlinear anisotropic diffusion}
But we can still go one step further and even make the direction of the smoothing process dependent
on the local structure. For example, one might want to prevent smoothing across edges and rather
smooth parallel to them further embracing the edge-preserving nature of nonlinear diffusion.

The diffusion tensor in this case is constructed from its eigenvectors $\boldsymbol v_1,
\boldsymbol v_2$ and eigenvalues $\lambda_1, \lambda_2$ using the
theorem of matrix diagonalisation
\begin{equation}
    \boldsymbol D = \left(\boldsymbol v_1\vert \boldsymbol v_2\right) \textnormal{diag}(\lambda_1, \lambda_2)
    \begin{pmatrix}
        \boldsymbol v_1^\top\\
        \boldsymbol v_2^\top
    \end{pmatrix}
\end{equation}
We define the the eigenvectors to be
\begin{align}
    \boldsymbol v_1 &\Vert \boldsymbol\nabla u_\sigma &\boldsymbol v_2 &\bot \boldsymbol\nabla
    u_\sigma
\intertext{As mentioned before, we want to encourage smoothing along edges and in flat regions. Thus, we
        define the eigenvalue for the eigenvector parallel to the gradient to be 1 and the eigenvalue to
        the orthogonal one to be a diffusivity function similar to the one in the nonlinear isotropic case.}
    \lambda_1 &= g(\lVert\boldsymbol\nabla u_\sigma\rVert_2^2) &\lambda_2 &= 1
\end{align}
There are different viable choices for a diffusivity function that supports backwards diffusion.
First, there is the already mentioned Perona-Malik diffusivity \eqref{def:Diffusivity}. Another
option is the diffusivity function introduced by Weickert\cite{weickert96}
\begin{equation}
    g(s^2) = \begin{cases}
        1 & s^2 = 0\\
        1 - \textnormal{exp}\left(\frac{-3.31488}{(s/\lambda)^8}\right) & s^2 > 0
    \end{cases}\label{def:WeickertDiff}
\end{equation}
According to \cite{dic18-04}, diffusivities of this type, i.e. rapidly decreasing diffusivities,
lead to more segmentation-like results.
Finally, the diffusivity that was mainly used for inpainting in this work is the
\textit{Charbonnier diffusivity} \cite{charbonnier}
\begin{equation}
    g(s^2) = \frac{1}{\sqrt{1 + s^2 / \lambda^2}}\label{def:CharbonnierDiff}
\end{equation}
\begin{figure}
    \centering
    \includegraphics[width=\linewidth]{../Images/diffusivities.png}
    \caption{Weickert \eqref{def:WeickertDiff} and Charbonnier \eqref{def:CharbonnierDiff} diffusivities for $\lambda = 1$. Graph created with Matplotlib}
\end{figure}
\begin{figure}
    \centering
    \includegraphics[width=0.8\linewidth]{../Images/diff_examples.png}
    \caption{Denoising capabilities of the different types of diffusion. \textbf{Top Left:} Original
    image. \textbf{Top Right:} Linear isotropic diffusion. \textbf{Bottom Left:} Nonlinear
isotropic diffusion. \textbf{Bottom Right:} Edge-enhancing diffusion. Source:
\cite{weickert96}}\label{fig:DiffExamples}
\end{figure}
%As we see in figure \ref{fig:DiffExamples}, EED possesses great restorative and denoising
%capabilities. But as Galić et al. showed, EED is also an excellent choice for
%PDE-based inpainting\cite{galic05}.
\section{EED-based inpainting}
Inpainting in digital image processing has first been introduced in \cite{bertalmio00}, as
mentioned in related work. They used 
\TODO{Shortly explain "normal" inpainting}
\TODO{Explain basics of EED inpainting}

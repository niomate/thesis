\chapter{Introduction}\label{ch:Intro}
As technology evolves, the quality and resolution of digital images improve as well. But as the
quality increases so does the memory required to store the image on a hard drive. To counteract
this increase in disk space usage, people have tried to reduce the sizes of digital images a lot in the
last decades.\\
One of the most successful and probably most well known \textit{codecs}
is \textbf{JPEG} and its successor \textbf{JPEG 2000}. Both are lossy image compression methods
known for fairly high compression rates while still providing a reasonably image quality.\\
For
higher compression rates however, the quality deteriorates pretty quickly and the infamous ``block
artifacts'' are being introduced. As a remedy, a new method for image compression has been developed in the last years that
aims to create better looking images for higher compression rates than JPEG and even JPEG2000. \\
This new method roughly works by selecting a small amount of pixels to keep and then filling in
the gaps in the reconstruction/decompression step.\\
As one can imagine, selecting the right data is a fairly minute process and one has to carefully
select the pixels to keep. Even though there has been a lot of work done in this area, the
selection can still be improved.\\
In the past, the usefulness of corners for this process was proven in~\cite{zimmer07} even though the
method proposed in this work would not surpass JPEG's abilities. Nonetheless, we want to build on
it and explore
how keeping larger regions of data around corners plays out in this process.

%%%%%%%%%%%%%%%%%%%%%%%%%%%%%%%%%%%%%%%%%;
% Masters/Doctoral Thesis
% LaTeX Template
% Version 2.5 (27/8/17)
%
% This template was downloaded from:
% http://www.LaTeXTemplates.com
%
% Version 2.x major modifications by:
% Vel (vel@latextemplates.com)
%
% This template is based on a template by:
% Steve Gunn (http://users.ecs.soton.ac.uk/srg/softwaretools/document/templates/)
% Sunil Patel (http://www.sunilpatel.co.uk/thesis-template/)
%
% Template license:
% CC BY-NC-SA 3.0 (http://creativecommons.org/licenses/by-nc-sa/3.0/)
%
%%%%%%%%%%%%%%%%%%%%%%%%%%%%%%%%%%%%%%%%%

%----------------------------------------------------------------------------------------
%   PACKAGES AND OTHER DOCUMENT CONFIGURATIONS
%----------------------------------------------------------------------------------------


\documentclass[
12pt, % The default document font size, options: 10pt, 11pt, 12pt
oneside, % Two side (alternating margins) for binding by default, uncomment to switch to one side
english, % ngerman for German
onehalfspacing, % Single line spacing, alternatives: onehalfspacing or doublespacing
%draft, % Uncomment to enable draft mode (no pictures, no links, overfull hboxes indicated)
%nolistspacing, % If the document is onehalfspacing or doublespacing, uncomment this to set spacing in lists to single
%liststotoc, % Uncomment to add the list of figures/tables/etc to the table of contents
toctotoc, % Uncomment to add the main table of contents to the table of contents
%parskip, % Uncomment to add space between paragraphs
%nohyperref, % Uncomment to not load the hyperref package
headsepline, % Uncomment to get a line under the header
chapterinoneline, % Uncomment to place the chapter title next to the number on one line
%consistentlayout, % Uncomment to change the layout of the declaration, abstract and acknowledgements pages to match the default layout
]{MastersDoctoralThesis} % The class file specifying the document structure

\usepackage[utf8]{inputenc} % Required for inputting international characters
\usepackage[T1]{fontenc} % Output font encoding for international characters
\usepackage{amsmath, amssymb}
\usepackage{babel}
\usepackage{pdfpages}
%\usepackage{mathpazo} % Use the Palatino font by default

\usepackage[maxbibnames=99, backend=biber, sorting=none]{biblatex} % Use the bibtex backend with the authoryear citation style (which resembles APA)
\addbibresource{references.bib}

\usepackage[autostyle=true]{csquotes} % Required to generate language-dependent quotes in the bibliography
\usepackage{listings}
\usepackage{subcaption}
\usepackage{xcolor}
\usepackage{colortbl}
\usepackage{diagbox}
\usepackage{array}
\usepackage{float}
\usepackage[a4]{crop}
%\usepackage[colorlinks=false]{hyperref}
%\usepackage{hyperref}

\graphicspath{{./Images/}{./images/}}

\crop

\substitutecolormodel{rgb}{cmyk}

% To get rid of widows and clubs.
\clubpenalty=10000
\widowpenalty=10000
\displaywidowpenalty=10000



%----------------------------------------------------------------------------------------
%   LISTING SETTINGS
%----------------------------------------------------------------------------------------

\definecolor{codegreen}{rgb}{0,0.6,0}
\definecolor{codegray}{rgb}{0.5,0.5,0.5}
\definecolor{codepurple}{rgb}{0.58,0,0.82}
\definecolor{backcolour}{rgb}{0.95,0.95,0.92}
\lstdefinestyle{mystyle}{
    basicstyle=\ttfambackgroundcolor=\color{backcolour},
    commentstyle=\color{codegreen},
    keywordstyle=\color{magenta},
    numberstyle=\tiny\color{codegray},
    stringstyle=\color{codepurple},
    basicstyle=\ttfamily\footnotesize,
    breakatwhitespace=false,
    breaklines=true,
    captionpos=b,
    keepspaces=true,
    numbers=left,
    numbersep=5pt,
    showspaces=false,
    showstringspaces=false,
    showtabs=false,
    tabsize=2
}
\lstset{style=mystyle}

%----------------------------------------------------------------------------------------
%   MARGIN SETTINGS
%----------------------------------------------------------------------------------------

\geometry{
    paper=a4paper, % Change to letterpaper for US letter
    inner=2.5cm, % Inner margin
    outer=3.8cm, % Outer margin
    bindingoffset=.5cm, % Binding offset
    top=1.5cm, % Top margin
    bottom=1.5cm, % Bottom margin
}

% Custom commands ---------------------------------------------------------------
\newcommand{\R}{\mathbb{R}}
\newcommand{\Z}{\mathbb{Z}}
\newcommand{\del}{\ensuremath{\mathbf{\nabla}}}
\DeclareMathOperator{\tr}{tr}
\DeclareMathOperator{\diverg}{div}
\DeclareMathOperator{\MSE}{MSE}
\DeclareMathOperator{\PSNR}{PSNR}


%----------------------------------------------------------------------------------------
%   THESIS INFORMATION
%----------------------------------------------------------------------------------------

\thesistitle{Exploring Disk-Shaped Corner Regions as Seed Points for PDE-based Inpainting} % Your thesis title, this is used in the title and abstract, print it elsewhere with \ttitle
\supervisor{Prof.\ Dr.\ Joachim Weickert} % Your supervisor's name, this is used in the title page, print it elsewhere with \supname
\examiner{Dr.\ Pascal Peter} % Your examiner's name, this is not currently used anywhere in the template, print it elsewhere with \examname
\degree{Bachelor of Science} % Your degree name, this is used in the title page and abstract, print it elsewhere with \degreename
\author{Jana Ariane Gusenburger} % Your name, this is used in the title page and abstract, print it elsewhere with \authorname

\subject{Computer Science} % Your subject area, this is not currently used anywhere in the template, print it elsewhere with \subjectname
\keywords{} % Keywords for your thesis, this is not currently used anywhere in the template, print it elsewhere with \keywordnames
\university{Universit\"at des Saarlandes} % Your university's name and URL, this is used in the title page and abstract, print it elsewhere with \univname
\department{Department of Computer Science} % Your department's name and URL, this is used in the title page and abstract, print it elsewhere with \deptname
\group{Mathematical Image Analysis Group} % Your research group's name and URL, this is used in the title page, print it elsewhere with \groupname
\faculty{Faculty of Mathematics and Computer Science} % Your faculty's name and URL, this is used in the title page and abstract, print it elsewhere with \facname

\AtBeginDocument{
    \hypersetup{pdftitle=\ttitle} % Set the PDF's title to your title
    \hypersetup{pdfauthor=\authorname} % Set the PDF's author to your name
    \hypersetup{pdfkeywords=\keywordnames} % Set the PDF's keywords to your keywords
}

\begin{document}

\frontmatter % Use roman page numbering style (i, ii, iii, iv...) for the pre-content pages

\pagestyle{plain} % Default to the plain heading style until the thesis style is called for the body content

%----------------------------------------------------------------------------------------
%   TITLE PAGE
%----------------------------------------------------------------------------------------

\begin{titlepage}
    \begin{center}
        {\scshape\LARGE \univname\par}\vspace{1cm} % University name
        \textsc{\Large Bachelor Thesis}\\[0.5cm] % Thesis type
        \hrule
        \vspace{0.5cm}
        {\huge \bfseries \ttitle\par}
        \vspace{0.5cm} % Thesis title
        \hrule
        \vspace{1.5cm}

        \begin{minipage}[t]{0.4\textwidth}
            \begin{flushleft} \large
                \emph{Author:}\\
                \authorname
            \end{flushleft}
        \end{minipage}
        \begin{minipage}[t]{0.5\textwidth}
            \begin{flushright} \large
                \emph{Supervisor/Advisor/Reviewer:} \\
                \supname\\[0.3cm]
                \emph{Second reviewer:}\\
                \examname\\
            \end{flushright}
        \end{minipage}\\[1cm]
        \large \textit{A thesis submitted in fulfillment of the requirements\\ for the degree of \degreename}\\[0.3cm] % University requirement text
        \textit{in the}\\[0.3cm]
        \groupname\\\deptname\\[1cm] % Research group name and department name
        \includegraphics[height=4cm]{Logo-Universität_des_Saarlandes.png} \\[1cm]
        \vfill
        {\large \today}
    \end{center}
\end{titlepage}

%----------------------------------------------------------------------------------------
%   DECLARATION PAGE
%----------------------------------------------------------------------------------------

\includepdf{Images/statement.pdf}

\cleardoublepage

\begin{acknowledgements}
    \addchaptertocentry{\acknowledgementname}
    Firstly, I want to thank Prof.\ Dr.\ Joachim Weickert for his support and professional insight
    on the topic as well as for providing me with an interesting topic for my thesis. He always
    had an open ear for technical questions even despite the difficult situation that was
    Covid-19.\\
\\
    \noindent Naturally, I also want to thank Dr.\ Pascal Peter for accepting my
    request to be the second reviewer on such fairly short notice.\\
\\
    \noindent Moreover, I want to express my gratitude towards my friends and family for being
    so supportive of me during the time of writing this thesis and for proof-reading early
    draft versions.\\
\\
    \noindent Last but certainly not least, I want to acknowledge my supervisor at work for
    giving me the necessary time off to finish my thesis.
\end{acknowledgements}

%----------------------------------------------------------------------------------------
%   ABSTRACT PAGE
%----------------------------------------------------------------------------------------

\begin{abstract}
    \addchaptertocentry{\abstractname}
    Over the last years, a new class of image compression codecs has been developed making use of
    novel inpainting techniques. These newly developed methods have been shown to have the
    potential to surpass common codecs like JPEG and JPEG2000. The idea behind them is to store a
    sparse subset of pixels and fill in the missing information with a digital image inpainting
    algorithm. The challenge with such an approach is the choice of pixels that serve as the basis
    to reconstruct the image. While the most successful ideas are based on stochastic or
    subdivision based methods, semantic approaches have barely been explored.
    In this thesis we build on results from a previous publication and further explore the potential
    of reconstructing images by using only a set of corners from the image. We examine how
    the accuracy of the corner detection influences the reconstruction quality and lastly introduce
    additional procedures that aim to make the mask calculation a bit more robust with respect to
    the pixel density of the inpainting mask.
\end{abstract}

%----------------------------------------------------------------------------------------
%   ACKNOWLEDGEMENTS
%----------------------------------------------------------------------------------------

%----------------------------------------------------------------------------------------
%   LIST OF CONTENTS/FIGURES/TABLES PAGES
%----------------------------------------------------------------------------------------

\tableofcontents % Prints the main table of contents

%\listoffigures % Prints the list of figures

%\listoftables % Prints the list of tables

%----------------------------------------------------------------------------------------
%   thesis content - chapters
%----------------------------------------------------------------------------------------

% begin numeric (1,2,3...) page numbering
\mainmatter

\pagestyle{thesis} % Return the page headers back to the "thesis" style

% Include the chapters of the thesis as separate files from the Chapters folder
% Uncomment the lines as you write the chapters

\include{Chapters/Introduction}
\include{Chapters/RelatedWork}
\include{Chapters/Theory}
\include{Chapters/Numerical}
\include{Chapters/Model}
\include{Chapters/Results}
\include{Chapters/Conclusion}

%----------------------------------------------------------------------------------------
%   THESIS CONTENT - APPENDICES
%----------------------------------------------------------------------------------------

\appendix % Cue to tell LaTeX that the following "chapters" are Appendices

%----------------------------------------------------------------------------------------
%   BIBLIOGRAPHY
%----------------------------------------------------------------------------------------

\printbibliography

%----------------------------------------------------------------------------------------

\end{document}

%%%%%%%%%%%%%%%%%%%%%%%%%%%%%%%%%%%%%%%%%;
% Masters/Doctoral Thesis
% LaTeX Template
% Version 2.5 (27/8/17)
%
% This template was downloaded from:
% http://www.LaTeXTemplates.com
%
% Version 2.x major modifications by:
% Vel (vel@latextemplates.com)
%
% This template is based on a template by:
% Steve Gunn (http://users.ecs.soton.ac.uk/srg/softwaretools/document/templates/)
% Sunil Patel (http://www.sunilpatel.co.uk/thesis-template/)
%
% Template license:
% CC BY-NC-SA 3.0 (http://creativecommons.org/licenses/by-nc-sa/3.0/)
%
%%%%%%%%%%%%%%%%%%%%%%%%%%%%%%%%%%%%%%%%%

%----------------------------------------------------------------------------------------
%   PACKAGES AND OTHER DOCUMENT CONFIGURATIONS
%----------------------------------------------------------------------------------------


\documentclass[
12pt, % The default document font size, options: 10pt, 11pt, 12pt
oneside, % Two side (alternating margins) for binding by default, uncomment to switch to one side
english, % ngerman for German
onehalfspacing, % Single line spacing, alternatives: onehalfspacing or doublespacing
%draft, % Uncomment to enable draft mode (no pictures, no links, overfull hboxes indicated)
%nolistspacing, % If the document is onehalfspacing or doublespacing, uncomment this to set spacing in lists to single
%liststotoc, % Uncomment to add the list of figures/tables/etc to the table of contents
toctotoc, % Uncomment to add the main table of contents to the table of contents
%parskip, % Uncomment to add space between paragraphs
%nohyperref, % Uncomment to not load the hyperref package
headsepline, % Uncomment to get a line under the header
chapterinoneline, % Uncomment to place the chapter title next to the number on one line
%consistentlayout, % Uncomment to change the layout of the declaration, abstract and acknowledgements pages to match the default layout
]{MastersDoctoralThesis} % The class file specifying the document structure

\usepackage[utf8]{inputenc} % Required for inputting international characters
\usepackage[T1]{fontenc} % Output font encoding for international characters
\usepackage{amsmath, amssymb}
\usepackage{babel}
%\usepackage{mathpazo} % Use the Palatino font by default

\usepackage[backend=biber, sorting=none]{biblatex} % Use the bibtex backend with the authoryear citation style (which resembles APA)
\addbibresource{references.bib}

\usepackage[autostyle=true]{csquotes} % Required to generate language-dependent quotes in the bibliography
\usepackage{listings}
\usepackage{subcaption} 
\usepackage{xcolor}
\usepackage{colortbl} 
\usepackage{diagbox}
\usepackage{array} 
\usepackage{float}

\graphicspath{{/home/daniel/Uni/Thesis/thesis/Images/}{/home/daniel/Uni/Thesis/images/}}


% To get rid of widows and clubs.
\clubpenalty=10000
\widowpenalty=10000
\displaywidowpenalty=10000

%----------------------------------------------------------------------------------------
%   LISTING SETTINGS
%----------------------------------------------------------------------------------------

\definecolor{codegreen}{rgb}{0,0.6,0}
\definecolor{codegray}{rgb}{0.5,0.5,0.5}
\definecolor{codepurple}{rgb}{0.58,0,0.82}
\definecolor{backcolour}{rgb}{0.95,0.95,0.92}
\lstdefinestyle{mystyle}{
    basicstyle=\ttfambackgroundcolor=\color{backcolour},   
    commentstyle=\color{codegreen},
    keywordstyle=\color{magenta},
    numberstyle=\tiny\color{codegray},
    stringstyle=\color{codepurple},
    basicstyle=\ttfamily\footnotesize,
    breakatwhitespace=false,         
    breaklines=true,                 
    captionpos=b,                    
    keepspaces=true,                 
    numbers=left,                    
    numbersep=5pt,                  
    showspaces=false,                
    showstringspaces=false,
    showtabs=false,                  
    tabsize=2
}
\lstset{style=mystyle}

%----------------------------------------------------------------------------------------
%   MARGIN SETTINGS
%----------------------------------------------------------------------------------------

\geometry{
    paper=a4paper, % Change to letterpaper for US letter
    inner=2.5cm, % Inner margin
    outer=3.8cm, % Outer margin
    bindingoffset=.5cm, % Binding offset
    top=1.5cm, % Top margin
    bottom=1.5cm, % Bottom margin
}

% Custom commands ---------------------------------------------------------------
\newcommand{\R}{\mathbb{R}}
\newcommand{\Z}{\mathbb{Z}}
\newcommand{\del}{\ensuremath{\mathbf{\nabla}}}
\DeclareMathOperator{\tr}{tr}
\DeclareMathOperator{\diverg}{div} 
\DeclareMathOperator{\MSE}{MSE} 
\DeclareMathOperator{\PSNR}{PSNR} 


%----------------------------------------------------------------------------------------
%   THESIS INFORMATION
%----------------------------------------------------------------------------------------

\thesistitle{Exploring Circular Corner Regions as Seed Points for PDE-based Inpainting} % Your thesis title, this is used in the title and abstract, print it elsewhere with \ttitle
\supervisor{Prof.\ Dr.\ Joachim Weickert} % Your supervisor's name, this is used in the title page, print it elsewhere with \supname
\examiner{Dr.\ Pascal Peter} % Your examiner's name, this is not currently used anywhere in the template, print it elsewhere with \examname
\degree{Bachelor of Science} % Your degree name, this is used in the title page and abstract, print it elsewhere with \degreename
\author{Daniel Gusenburger} % Your name, this is used in the title page and abstract, print it elsewhere with \authorname

\subject{Computer Science} % Your subject area, this is not currently used anywhere in the template, print it elsewhere with \subjectname
\keywords{} % Keywords for your thesis, this is not currently used anywhere in the template, print it elsewhere with \keywordnames
\university{Universit\"at des Saarlandes} % Your university's name and URL, this is used in the title page and abstract, print it elsewhere with \univname
\department{Department of Computer Science} % Your department's name and URL, this is used in the title page and abstract, print it elsewhere with \deptname
\group{Mathematical Image Analysis Group} % Your research group's name and URL, this is used in the title page, print it elsewhere with \groupname
\faculty{Faculty of Mathematics and Computer Science} % Your faculty's name and URL, this is used in the title page and abstract, print it elsewhere with \facname

\AtBeginDocument{
    \hypersetup{pdftitle=\ttitle} % Set the PDF's title to your title
    \hypersetup{pdfauthor=\authorname} % Set the PDF's author to your name
    \hypersetup{pdfkeywords=\keywordnames} % Set the PDF's keywords to your keywords
}

\begin{document}

\frontmatter % Use roman page numbering style (i, ii, iii, iv...) for the pre-content pages

\pagestyle{plain} % Default to the plain heading style until the thesis style is called for the body content

%----------------------------------------------------------------------------------------
%   TITLE PAGE
%----------------------------------------------------------------------------------------

\begin{titlepage}
    \begin{center}
        {\scshape\LARGE \univname\par}\vspace{1cm} % University name
        \textsc{\Large Bachelor Thesis}\\[0.5cm] % Thesis type
        \hrule
        \vspace{0.5cm}
        {\huge \bfseries \ttitle\par}
        \vspace{0.5cm} % Thesis title
        \hrule
        \vspace{1.5cm}

        \begin{minipage}[t]{0.4\textwidth}
            \begin{flushleft} \large
                \emph{Author:}\\
                \authorname
            \end{flushleft}
        \end{minipage}
        \begin{minipage}[t]{0.5\textwidth}
            \begin{flushright} \large
                \emph{Supervisor/Advisor/Reviewer:} \\
                \supname\\[0.3cm]
                \emph{Second reviewer:}\\
                \examname\\
            \end{flushright}
        \end{minipage}\\[1cm]
        \large \textit{A thesis submitted in fulfillment of the requirements\\ for the degree of \degreename}\\[0.3cm] % University requirement text
        \textit{in the}\\[0.3cm]
        \groupname\\\deptname\\[1cm] % Research group name and department name
        \includegraphics[height=4cm]{Logo-Universität_des_Saarlandes.png} \\[1cm]
        \vfill
        {\large \today}
    \end{center}
\end{titlepage}

%----------------------------------------------------------------------------------------
%   DECLARATION PAGE
%----------------------------------------------------------------------------------------

\begin{declaration}
    \addchaptertocentry{Declaration of Authorship}
    \noindent I declare that this thesis titled, \enquote{\ttitle} and the work 
    presented in it are my own. I confirm that this thesis is my own work and that I have
    documented all sourced used.\\

    \noindent Signed:\\
    \rule[0.5em]{25em}{0.5pt} % This prints a line for the signature

    \noindent Date:\\
    \rule[0.5em]{25em}{0.5pt} % This prints a line to write the date
\end{declaration}

\cleardoublepage

\begin{acknowledgements}
    \addchaptertocentry{\acknowledgementname}
    Firstly, I want to thank Prof.\ Dr.\ Joachim Weickert for his support and professional insight
    on the topic as well as for providing me with an interesting topic for my thesis. He always
    had an open ear for technical questions even despite the difficult situation that was
    Covid-19.\\
\\
    \noindent Naturally, I also want to thank Dr.\ Pascal Peter for accepting my
    request to be the second reviewer on such fairly short notice.\\
\\
    \noindent Moreover, I want to express my gratitude towards my friends and family for being 
    so supportive of me during the time of writing this thesis and for proof-reading early
    draft versions.\\
\\
    \noindent Last but certainly not least, I want to acknowledge my supervisor at work for 
    giving me the necessary time off to finish my thesis.
\end{acknowledgements}

%----------------------------------------------------------------------------------------
%   ABSTRACT PAGE
%----------------------------------------------------------------------------------------

\begin{abstract}
    \addchaptertocentry{\abstractname} 
    Over the last years, a new class of image compression codecs has been developed making use of
    novel inpainting techniques. These newly developed methods have been shown to have the
    potential to surpass common codecs like JPEG and JPEG2000. The idea behind them is to store a
    sparse subset of pixels and fill in the missing information with a digital image inpainting 
    algorithm. The challenge with such an approach is the choice of pixels that serve as the basis 
    to reconstruct the image. While the most successful ideas are based on stochastic or
    subdivision based methods, semantic approaches have barely been explored.
    In this thesis I build on results from a previous publication and further explore the potential 
    of reconstructing images by using only a set of corners from the image. I examine how 
    the accuracy of the corner detection influences the reconstruction quality and lastly introduce 
    additional procedures that aim to make the mask calculation a bit more robust with respect to
    the pixel density of the inpainting mask.
\end{abstract}

%----------------------------------------------------------------------------------------
%   ACKNOWLEDGEMENTS
%----------------------------------------------------------------------------------------

%----------------------------------------------------------------------------------------
%   LIST OF CONTENTS/FIGURES/TABLES PAGES
%----------------------------------------------------------------------------------------

\tableofcontents % Prints the main table of contents

%\listoffigures % Prints the list of figures

%\listoftables % Prints the list of tables

%----------------------------------------------------------------------------------------
%   thesis content - chapters
%----------------------------------------------------------------------------------------

% begin numeric (1,2,3...) page numbering
\mainmatter

\pagestyle{thesis} % Return the page headers back to the "thesis" style

% Include the chapters of the thesis as separate files from the Chapters folder
% Uncomment the lines as you write the chapters

\chapter{Introduction}\label{ch:Intro}
As technology evolves, the quality and resolution of digital images improve as well. But as the
quality increases so does the memory required to store the image on a hard drive. To counteract
this increase in disk space usage, people have tried to reduce the sizes of digital images a lot in the
last decades.\\
One of the most successful and probably most well known \textit{codecs}
is \textbf{JPEG} and its successor \textbf{JPEG 2000}. Both are lossy image compression methods
known for fairly high compression rates while still providing a reasonably image quality.\\
For
higher compression rates however, the quality deteriorates pretty quickly and the infamous ``block
artifacts'' are being introduced. As a remedy, a new method for image compression has been developed in the last years that
aims to create better looking images for higher compression rates than JPEG and even JPEG2000. \\
This new method roughly works by selecting a small amount of pixels to keep and then filling in
the gaps in the reconstruction/decompression step.\\
As one can imagine, selecting the right data is a fairly minute process and one has to carefully
select the pixels to keep. Even though there has been a lot of work done in this area, the
selection can still be improved.\\
In the past, the usefulness of corners for this process was proven in~\cite{zimmer07} even though the
method proposed in this work would not surpass JPEG's abilities. Nonetheless, we want to build on
it and explore
how keeping larger regions of data around corners plays out in this process.

\chapter{Related Work}\label{ch:RelatedWork}


\section{PDE-based image compression}

In recent years, a new image codec based on partial differential equations was developed as an
alternative to JPEG and JPEG2000. The core idea is to compress the image by selecting a set of
pixels that serves as the basis for an inpainting process. This inpainting process is defined by a
partial differential equation that is solved iteratively during the reconstruction phase.\\
In 2005, Galić et al.\cite{galic05}\ first introduced an alternative image compression method using nonlinear
anisotropic diffusion
 as
a serious alternative to more classical approaches like JPEG. In this
work, the authors showed the inpainting capabilities of a diffusion process known as edge-enhancing
diffusion, which has since then established itself as the prime choice for PDE-based image
compression. The approach they presented relies on efficiently storing a pixel mask that is later
on used to reconstruct the image by filling in the missing regions using the aforementioned process
of edge-enhancing diffusion and lays the groundwork for later publications building on the codec
defined herein. \\
As already mentioned, the codec relies on computing a pixel mask from the original image. However,
this process is a balancing act. On one hand, one wants to get a mask that yields a perfect
reconstruction of the original image but on the other hand, one also wants to create a mask that
can be encoded and stored efficiently, i.e. using the smallest amount of \textit{bits-per-pixel
(bpp)} possible. This step of mask computation is still topic of ongoing research which proves the
difficulty of this problem.\\
In the initial codec, also called the \textbf{BTTC-EED} codec, the mask was
computed by means of an adaptive sparsification scheme relying on B-tree triangular coding (BTTC)
that was already introduced by Distasi et al\cite{distasi97} back in 1997. This fairly simple
approach iteratively subdivides the image diagonally if the error of the reconstruction of the image using only
the corner points of the subdivision exceeds an a priori defined threshold. In this version, the
reconstruction is approximated by a simple linear interpolation inside the triangle. The efficiency
of constructing a mask using BTTC lies in the binary tree structure of the subdivision that can be encoded extremely
efficiently by a simple binary string. Furthermore, grey values are encoded using a straight
forward entropy
coding method like Huffman coding\cite{huffman}.
With this fairly simple approach they were already able to outperform JPEG visually for high
compression rates and comic-style images~\cite{galic05}.\\
In 2008, the same authors improved this version by adding a requantisation step to the encoding
phase, in which they quantised the grey values form 256 values to 64 in order to further shorten
the sequences obtained by Huffman coding.

%Improving on this, the authors published a new paper in 2008, adding a number of additional
%procedures to the compression phase, with which they were finally able to come close to the quality
%of JPEG 2000~\cite{galic08}.
%Finally, in 2009, Schmaltz et al.\ optimised the ideas even further, building the
%so called \textbf{R-EED} codec with which they could even beat JPEG 2000~\cite{schmaltz09}.
%The main differences between~\cite{galic05} and~\cite{schmaltz09} are the addition of several
%procedures to optimise the data set that is kept for inpainting in the decompression step.
%To roughly summarise the whole compression phase~\cite{schmaltz09}:\\
%First, an initial set of points is gathered by using a rectangular subdivision (instead of the
%previous triangular subdivision) of the image. This works by recursively splitting the image in
%half whenever the reconstruction using only the boundary points exceeds a certain error threshold.
%The reconstruction is also done using EED inpainting.
%After obtaining the initial data set, the brightness values of each of the kept pixels is rescaled
%to $[0, 255]$ to eliminate possible quantisation artifacts. Due to this brightness rescaling, the
%optimal contrast parameter for the decompression phase may change and thus has to be adjusted as
%well. This is generally done alternating between optimising points and the contrast
%parameter until a certain convergence criterion is met.
%Abs a last step, the authors invert the inpainting mask and perform the inpainting process to fill in
    %the kept data as a means to increase the coherence between the optimised pixels and the
    %original image.

    %All of these measures serve the purpose of decreasing the \textit{mean squared error (MSE)} to a
%level where the proposed codec is able to outperform JPEG 2000 for compression rates higher than
%$\mathbf{43:1}$.

\section{Image features in image compression}
Features such as edges and corners are very important in the field of image processing as they
provide almost all of the semantics of an image\cite{marr82}. Therefore, feature detection has been a staple in
this field for a long time. Actually, corner and edge detection algorithms such as the ones by John Canny
and Chris Harris\cite{harris88, canny86} are still used today. 
Despite their semantical importance, edges and corners have not had that much impact on image
compression so far.\\

In 2007, Zimmer introduced a new approach using corners to compute inpainting masks for
PDE-based image compression\cite{zimmer07}. In their approach, they used a simple corner detection method to
sample a set of the most important corners. The inpainting mask was then obtained by storing the
8-neighbourhood around each corner. In the reconstruction/inpainting phase they used a method
following the idea proposed by Bertalmio et al\cite{bertalmio00}. Instead of just using pure EED to
inpaint the image, they interleaved edge-enhancing diffusion with mean curvature motion in order to
improve the inpainting in regions with sparser inpainting domains. \\
Even though their codec was
not able to compete with widely used codecs like JPEG and JPEG2000 they proved that corners are
indeed useful for PDE based inpainting. They found the largest disadvantages to be the sparsity of
corners in images. The reasoning behind this is that because of the sparsity of corners in an
image, one has to either invest in a very sophisticated and complex inpainting mechanism to fill in
the large areas between the few detected corners \textit{or} adjust the corner detector to be more
`fuzzy'. On one hand, this leads to a denser mask and hence better results reconstruction-wise 
but on the other also creates suboptimal inpainting masks. Due to the fuzzy nature of the
corner detector, flat or homogeneous areas are classified as corners and therefore kept in the
mask, even though these regions could have easily been filled in by even a simple inpainting
process like linear diffusion. \\
To improve on their work, they suggested to touch on the parameter
selection for the corner detector, especially the selection of both the cornerness threshold and
integration scale used in the computation of the structure tensor since these heavily influence the
amount and quality of corners that are detected. As another improvement, they suggested different
shapes of corner regions. My thesis is largely built upon the results from this work and tries to
implement the improvement ideas proposed by Zimmer to see how well they work.\\

In \cite{mainberger09, mainberger10}, the authors implemented a diffusion based reconstruction
using mainly edges as the inpainting mask. For the mask construction they used the Marr-Hildreth
edge detector combined with \textit{hysteresis thresholding} as proposed by Canny\cite{canny86}.
They stated however, that they were not locked in on the Marr-Hildreth edge detector and that
others such as the classical Canny edge detector could also be used, especially for images that
contain a larger amount of blurry edges. The addition of hysteresis thresholding yielded closed and
well localised contours which they encoded using the lossless \textit{JBIG} encoding\cite{jbig}.
To reconstruct the image from the stored contours, they used a simple linear diffusion approach
which, in this case, was good enough. For cartoon-like images, they could beat the quality of
JPEG and even the more sophisticated JPEG2000 in terms of the PSNR (peak signal to noise ratio) of the reconstructed image to the
original one. This was all done in the earlier work\cite{mainberger09}. In the follow-up
publication from the same authors in 2010 they introduced some optimisations such as different
entropy coders for the edge locations as well as an optimised method of storing the grey/colour
values of the mask pixels. Another improvement was the use of an advanced method for solving the
linear systems arising from the inpainting problem. In contrast to the earlier publications a fast
\textit{full multigrid scheme} was implemented to speed up the decoding phase significantly to 
a point where the codec is now realtime capable.\\

In \cite{peter15}, region of interest(ROI) coding was first introduced for the R-EED codec. Previously,
it was not possible to specify certain regions of an image to be reconstructed with higher detail.
With this new contribution, they allowed the user to specify a weighting mask that is used during
the mask construction phase to adapt the mask in such a way that the specified regions would have a
denser mask than other regions. This is expecially useful in medical imaging, where some regions of
an image, e.g. in a CT scan of a brain region, need to be reconstructed exactly in order to prevent
false diagnoses. In this work however, the weight masks had to be chosen manually, which is not
always very practical (cf. section \ref{sec:Discussion}: Discussion).
\\
ROI coding was not the only contribution in this publication. The authors also proposed
\textit{progressive modes} for the R-EED codec to be able to compete better with widely available codecs.
Progressive modes allow an image to be displayed in a coarse manner, e.g. when shown on a website
that has not fully loaded the image data yet, and gradually refine the representation as more data
becomes available. Last but certainly not least, they showed that their PDE-based codec is also
capable of realtime video decoding and playback which sets a milestone on the way to prove the
\textit{suitability of R-EED for real-world applications}\cite{peter15}. In their words this
publication, or rather the extensions presented therein, mark the first step of an evolution of
PDE-based compression codecs from the proof-of-concept stage to fully grown codecs with relevance
for practical applications.\cite{peter15}\\

\section{Outline}
So far, we have reviewed some of the work that is closely related and motivated the goal of our work.
To explain what exactly was done in order to get to this goal, we first have to go over some of the
theoretical background. We will introduce basic concepts from
calculus and linear algebra (\ref{sec:Basics}) as well as some more advanced topics
like the structure tensor (\ref{sec:Structure}) and
inpainting methods (\ref{sec:Inpainting}).\\
In chapter \ref{ch:Implementation} we will then talk about the technical details of our 
implementation, for example the discretisation used to implement the theoretical ideas from 
the previous chapter (\ref{sec:Discretisation}). 
Furthermore, we will present some additions we made and the reasoning behind them
(\ref{sec:Contribution}).\\
Afterwards, we discuss the experiments that were concluded in order to evaluate the initial idea as
well as show the results these experiments yielded.\\
Last but not least in chapter \ref{ch:Conclusion}, we conclude our work, discuss the strengths
and shortcomings of our approach, h ow it could see some improvements and where it could be
applied to.\\


\chapter{Theory}\label{ch:Theory}

\section{Basics}\label{sec:Basics}
The concepts used in this thesis require some prior knowledge about basic calculus and linear
algebra as well as some more advanced topics that will be introduced in the following sections.
But before introducing corner detection and diffusion, we have to first define what an image is
mathematically.\\
A \textit{grey value image} is defined as a function $f: \Omega \rightarrow [0, 255]$ where
$\Omega \subset \mathbb{R}^2$ is a rectangular subset of $\mathbb{R}^2$ of size $n_x\times n_y$,
wheras a
\textit{colour image} is defined as a vector-valued function $f: \Omega \rightarrow [0, 255]^3$.
For the sake of simplicity, we will focus on grey value images as most of the results can easily be
transferred to vector-valued images.

\subsection{Image gradient}
One of the most important operations on functions in Image Processing is \textit{partial
    differentiation}.
The partial derivative of an image $f: \Omega \rightarrow [0, 255]$ in $x$-direction is herein denoted as $f_x$ or
synonymously as $\partial_x f$ and defined as
\begin{equation}
    f_x(x, y) = \partial_x f (x, y) = \frac{\partial f}{\partial x} (x, y) := \lim_{h \to 0}\frac{f(x+h, y) -f(x, y)}{h} 
\end{equation}
The \textit{gradient} of an image $f$ is the vector containing both partial image derivatives.
In multivariable calculus, the gradient of a function is an important tool to find the (both local
and global) extrema of a function similar to the first derivative for a function with a single
variable.
\begin{equation}
    \textbf{grad}(f) = \boldsymbol\nabla f := \left(f_x, f_y\right)^\top
\end{equation}
Note that the gradient of a function is a vector-valued function and not a vector.
More intuitively, the gradient determines the direction and value of the fastest decrease of the
given function.

\subsection{Convolution}
Another operation from calculus that we will need is the \textit{convolution operator}.\\
\begin{align}
    (f * g)(x) &:= \int\limits_\mathbb{R} f(x-x')g(x')dx'\label{eq:1DConv}\\
    (f * g)(x, y) &:= \iint\limits_\mathbb{R} f(x-x', y-y')g(x', y')dx'dy'\label{eq:2DConv}
\end{align}
Convolution is especially useful in Image and Signal Processing to design so called linear filters
such as a moving average or smoothing operation\cite{ipcv19-02}\cite{dspguide}. As a matter of fact, in a later
section we will need the convolution as a tool to smooth our image to reduce noise artifacts. To
achieve this, we will use a \textit{Gaussian convolution}, i.e.\ a convolution with a
\textit{Gaussian kernel} which is basically just a two-dimensional Gaussian function with a certain
standard deviation\cite{ipcv19-02}:
\begin{equation}
    K_\sigma (x, y) := \frac{1}{2\pi\sigma^2}\exp\left(\frac{-(x-\mu_1)^2 - (y -
            \mu_2)^2}{2\sigma^2}\right)
\end{equation}
In the definition above, $\mu_1, \mu_2$ are the mean values in each direction.
For the rest of this thesis, an image $f$ convolved with a Gaussian with standard deviation $\sigma$
will be denoted by \[f_\sigma := K_\sigma * f\]
Note that because of the symmetry of the
convolution, it would have been perfectly fine to write it as $f * K_\sigma$. If the reader wants
to know more about convolution and its properties, they are kindly referred to \cite{dspguide}.

\section{The Structure Tensor}\label{sec:Structure}
For some applications only the gradient of an image does not give us enough information. The
gradient on its own is mostly just used as an edge detector, hence we need to come up with
something else for e.g.\ corner detection\cite{ipcv19-13}. One option is the so called \textit{structure tensor}, a
matrix that contains information about the surrounding region at a specific position. With the
structure tensor, or rather its eigenvalues (cf. section \ref{sec:Corner}), one is able to
distinguish between flat regions, edges and corners.
\subsection{Definition}
The structure tensor is defined as a matrix whose eigenvectors tell us the direction of
both the largest and smallest grey value change. Mathematically, we can model this
as an optimisation problem:\\
Let $u$ be a grey value image.
We want to find a unit vector $\mathbf{n} \in \mathbb{R}^2$ that is `most parallel' or `most orthogonal' to the
gradient $\boldsymbol\nabla u$ within a circle of radius $\rho > 0$, i.e. one wants to optimise the
energy function\cite{harris88,ipcv19-13}.
\begin{align}
    E(\mathbf{n}) &= \int\limits_{B_\rho(x, y)} \left(\mathbf{n}^\top\boldsymbol\nabla u\right)^2dx'dy'\\
    &= \mathbf{n}^\top \left(\int\limits_{B_\rho(x, y)} \boldsymbol\nabla u \boldsymbol\nabla
        u^\top dx'dy' \right) \mathbf{n}
\end{align}
This unit vector is one of the eigenvectors of the matrix 
\[M_\rho(\boldsymbol\nabla u) := \int\limits_{B_\rho(x, y)} \boldsymbol\nabla u \boldsymbol\nabla
    u^\top dx'dy = b_\rho * (\boldsymbol\nabla u\boldsymbol\nabla u^\top)\]
where
\[b_\rho(x, y) = \begin{cases} 1 & x^2 + y^2 \leq \rho^2\\ 0 & \text{else} \end{cases}\]
is the indicator function of the circle $B_\rho$\cite{ipcv19-13} (in this case, the convolution is
applied componentwise). One can easily prove that $\mathbf{n}$ is indeed an
eigenvector of $M_\rho(\boldsymbol\nabla u)$. 
Let $\lambda = E(\mathbf{n})$, then
\begin{align}
    \mathbf{n}^\top M_\rho(\boldsymbol\nabla u) \mathbf{n} &= \lambda
    \intertext{Multiplication with $\mathbf{n}$ on both sides yields}
    M_\rho(\boldsymbol\nabla u)\mathbf{n} &= \lambda\mathbf{n}
\end{align}
And as long as $\mathbf{n}$ is non-zero, $\mathbf{n}$ is thus by definition an eigenvector of
$M_\rho(\boldsymbol\nabla u)$.\hfill$\blacksquare$\\
However, one should always use a gaussian window function with standard deviation $\rho$ instead of
the aforementioned indicator function simply for smoothness reasons\cite{harris88,ipcv19-13}. $\rho$ is also
called the \textit{integration scale} and determines how localised the structure information
is\cite{ipcv19-13}.
The lower the integration scale, the more it is focussed on the central point.
This leads to the definition
\begin{equation}
    \mathbf{J}_\rho(\boldsymbol\nabla u) := K_\rho * (\boldsymbol\nabla u\boldsymbol\nabla u^\top)
\end{equation}
To keep equations simpler, I will omit the brackets and just simply use $\mathbf{J}_\rho$ as the
structure tensor.
\subsection{Properties}
The structure tensor is a symmetric matrix and thus possesses orthonormal eigenvectors $\bold v_1,
\bold v_2$ with real-valued eigenvalues $\lambda_1, \lambda_2 \geq 0$. \cite{ipcv19-13} As
mentioned in the preface to this section, we can use these eigenvalues to distinguish between
corners, edges and flat regions as seen in figure \ref{fig:Structure}.
\begin{figure}
    \centering
    \includegraphics[height=0.25\pdfpageheight]{../Images/structure_tensor.png}
    \caption{Visualisation of distinction of image features using the eigenvalues of the structure
        tensor.\cite{harris88}}\label{fig:Structure}
\end{figure}
\subsection{Usage in Corner Detection}
\section{Diffusion}\label{sec:Diffusion}
\subsection{What is Diffusion?}
\subsection{Linear Diffusion}
\subsection{Nonlinear Diffusion}\label{sub:NonlinearDiff}
\section{Basics of Inpainting}\label{sec:Inpainting}
\subsection{EED-based Inpainting}

\chapter{Implementation}\label{ch:Implementation} 
In this chapter, I will present my contributions to this work and go over the technical side of
things. First off, we need to discretise the theory introduced in chapter \ref{ch:Theory} in order
to implement the corner detection and inpainting algorithms. To do that, we need to talk about
discrete images and the method of finite differences to approximate image derivatives.
After that we shortly talk about discretisation of diffusion processes which we need to implement
the inpainting/restoration part to test the mask we chose.\\  
In the second chapter I will introduce corner regions as an essential concept to this work and talk
about the additions we made to the corner detection algorithm in order to select the most valuable
corners for our inpainting mask.\\
All the source code is written in pure C and can be found in the appendix.
The initial corner detection algorithm as well as the code for the inpainting alogorithm was given
to me courtesy of Joachim Weickert.

\section{Discretisation}\label{sec:Discretisation}

Since reality is not infinitely fine, things such as infinitesimal calculations as seen in 
calculus, e.g. differentation of functions, can not be applied to the real world directly.
This is a problem, because digital images are inherently not continuous as they contain only a
finite number of pixels. 
We could have also solved the theoretical problem in a discrete domain but that would have 
been much more troublesome.
That is why we rather develop a continuous theory and then discretise it later to actually
implement the ideas as algorithms.

\subsection{Discrete images}
Let $f:\Omega \rightarrow \mathbb{R}$ be an image where $\Omega
:= (0, n_x)\times(0, n_y) \subset \mathbb{R}^2$ as defined in section \ref{sec:Basics}. To
\textit{sample} the image, i.e. to discretise the image domain, we assume that all pixels lie on a
rectangular equidistant grid inside $\Omega$, where each cell in the grid has a size of $h_x
\times h_y$.
That yields $N_x := n_x/h_x$ pixels in x- and $N_y := n_y/h_y$ pixels in
y-direction.
That being said, we define the pixel $u_{i,j}$ at grid location $(i, j)^\top$ as
\begin{equation}
    u_{i, j} := u(ih_x, jh_y)\qquad \forall(i ,j) \in \{1,\dots,N_x\}\times\{1,\dots,N_y\}
\end{equation}
With that approach, the pixels are defined to lie on the crossing of the grid lines.
An alternative idea defines the pixels to lie in the centre of each cell, i.e. at location 
$((i-\frac{1}{2})h_x,\ (j- \frac{1}{2})h_y)^\top$.
As a sidenote, the cell sizes in either direction are pretty much always assumed to be 1 in 
practice. 
But to keep the theory as universal as possible, we will use $h_x$ and $h_y$ instead.\\
Sampling of the spatial domain is not the only step necessary to fully discretise an image. We also have
to discretise the \textit{co-domain} or \textit{grey-value-domain}. In theory our grey value domain
is just $\mathbb{R}$, but since this is rather unpractical, we limit it to $[0, 255]$. This step of
the discretisation process is also called \textit{quantisation}.

\subsection{Numerical differentiation}

Image derivatives are essential to image processing as seen in the previous chapter. Therefore we
need a way to compute them even on discrete images. To compute the gradient or in the simpler case
just the derivative of a discrete function, one generally uses so called \textit{finite difference
schemes}. Such a scheme is normally derived from the \textit{Taylor expansion} of the continuous
function. For example, we want to compute the first derivative of a 1D function $f:\mathbb{R}
\rightarrow \mathbb{R}$.
The Taylor expansion of \textit{degree $n$} of this function around the point $x_0\in\mathbb{R}$ is given by 
\begin{equation}
    f(x) = T_n(x, x_0) + \mathcal{O}(h^{n+1})
\end{equation}
where $\mathcal{O}(h^{n+1})$ describes the magnitude of the leading error term and as such the
\textit{approximation quality} of the Taylor series.
The actual Taylor series is defined as
\begin{equation}
    T_n(x, x_0) = \sum\limits_{k=0}^{n} \frac{(x-x_0)^k}{k!}f^{(k)}(x_0)
    \footnote{$f^{(k)}$ denotes the $k$-th derivative of the function $f$}
\end{equation}
A finite difference scheme generally uses a weighted sum of neighbouring values to compute the
desired derivative expression. In our example, we want to derive a scheme to compute the first
derivative of $f_i$ using its neighbours $f_{i-1}$ and $f_{i+1}$, i.e.
\begin{equation}
    f_i' \approx \alpha f_{i-1} + \beta f_i + \gamma f_{i+1}
\end{equation}
We can now describe $f_{i-1}$ and $f_{i+1}$ in terms of their Taylor expansion around $f_i$: 
\begin{align}
    f_{i-1} &= f((i-1)h) \notag\\
            &= T_n((i-1)h, ih) + \mathcal{O}(h^{n+1})\notag\\
            &= \sum\limits_{k=0}^{n}\frac{(-h)^k}{k!}f_i^{(k)}+ \mathcal{O}(h^{n+1})\\
    f_{i+1} &= \dots = \sum\limits_{k=0}^{n}\frac{h^k}{k!}f_i^{(k)}+ \mathcal{O}(h^{n+1})
\end{align}
If we now choose a concrete value for $n$ (here $n=5$) we can actually compute the approximation:
\begin{align}
    f_{i-1} &= f_i - hf_i' + \frac{h^2}{2}f_i'' - \frac{h^3}{6}f_i''' + \frac{h^4}{24}f_i'''' -
    \frac{h^5}{120}f_i''''' + \mathcal{O}(h^6)\label{eq:fi-1}\\
    f_{i+1} &= f_i + hf_i' + \frac{h^2}{2}f_i'' + \frac{h^3}{6}f_i''' + \frac{h^4}{24}f_i'''' +
    \frac{h^5}{120}f_i''''' + \mathcal{O}(h^6)\label{eq:fi+1}
\end{align}
The next step is the \textit{comparison of coefficients}, we insert
\eqref{eq:fi-1} and \eqref{eq:fi+1} into the equation 
and solve the arising linear system of equations for $\alpha,\beta,\gamma$.
\begin{align}
    0\cdot f_i + 1\cdot f_i' + 0\cdot f_i'' \overset{!}{=} \alpha f_{i-1} + \beta f_i + \gamma
    f_{i+1}
\end{align}
After the substitution, the right hand side becomes
\begin{align}
    &\alpha \left(f_i - hf_i' + \frac{h^2}{2}f_i''\right) + \beta f_i + \gamma\left(f_i + hf_i' + \frac{h^2}{2}f_i''\right)\notag\\
    = &\left(\alpha+\beta+\gamma\right)f_i + h\left(-\alpha+\gamma\right)f_i' + \frac{h^2}{2}\left(\alpha+\gamma\right)f_i''
\end{align} 
Note that for the comparison of coefficients it suffices to use the first 3 summands of the
approximation.
The linear system defined by the above equation
\begin{equation}
    \begin{pmatrix}
        1&1&1\\
        -1&0&1\\
        1&0&1
    \end{pmatrix}
    \begin{pmatrix}
        \alpha\\
        \beta\\
        \gamma
    \end{pmatrix}
    =
    \begin{pmatrix}
        0\\
        \frac{1}{h}\\
        0
    \end{pmatrix}
\end{equation}
has the solutions $\alpha = -\frac{1}{2h}, \beta = 0, \gamma = \frac{1}{2h}$.
This yields the approximation 
\begin{equation}
    f_i'\approx\frac{f_{i+1} - f_{i-1}}{2h}
\end{equation}
To find out how good this scheme is, we re-insert \eqref{eq:fi-1} and \eqref{eq:fi+1} to get
\begin{align}
    \frac{f_{i+1} - f_{i-1}}{2h}= -&\frac{1}{2h}\left(f_i - hf_i' + \frac{h^2}{2}f_i'' - \frac{h^3}{6}f_i''' + \frac{h^4}{24}f_i'''' -
    \frac{h^5}{120}f_i''''' + \mathcal{O}(h^6)\right) + \notag\\
     &\frac{1}{2h}\left(f_i - hf_i' + \frac{h^2}{2}f_i'' - \frac{h^3}{6}f_i''' + \frac{h^4}{24}f_i'''' -
    \frac{h^5}{120}f_i''''' + \mathcal{O}(h^6)\right)\notag
\end{align}
Expanding and simplifying yields
\begin{align}
        \frac{f_{i+1} - f_{i-1}}{2h} &= f_i' + \underbrace{\frac{h^2}{6}f_i'' + \frac{h^4}{30}f_i'''' +
        \mathcal{O}(h^5)}_\text{quadratic leading error term}\notag\\
            \Rightarrow \frac{f_{i+1} - f_{i-1}}{2h} &= f_i' + \mathcal{O}(h^2)
\end{align}
This means that the error of our approximation is quadratic in the grid size. 
We also say that this approximation has a \textit{consistency order} of 2. Note that for such an
approximation to be reasonable, it has to have at least consistency order 1. Otherwise, it is
not guaranteed that the error term diminishes if we send the grid size $h$ to 0.\\
The scheme derived above is also called \textit{central difference scheme}. Not that there are
other schemes such as \textit{forward} and \textit{backward} differences, but for the most part, we
will only use central differences since it provides us with the highest consistency
order out of all three. 
\begin{align}
    f_i' &= \frac{f_i - f_{i-1}}{h} + \mathcal{O}(h)&\textnormal{(backward differences)}\notag\\
    f_i' &= \frac{f_{i+1} - f_i}{h} + \mathcal{O}(h)&\textnormal{(forward differences)}\notag
\end{align}
\subsection{Numerical schemes for diffusion}
\section{Corner regions}\label{sec:Contribution}
Our starting point for the detection of the most relevant corners was the classic Foerstner-Harris
corner detector based on the structure tensor (cf. section \ref{sub:Corner}). 
\begin{equation}
    \frac{\textnormal{det}(\boldsymbol J_\rho)}{\textnormal{tr}(\boldsymbol J_\rho)} =
    \frac{\lambda_1\lambda_2}{\lambda_1 + \lambda_2}\label{def:Harris}
\end{equation}
With that we build on the method that Zimmer \cite{zimmer07} examined in 2007. They chose the
Foerstner-Harris detector because it is very accurate and simulatneously not as computationally
invested as the Tomasi-Kanade detector that requires to compute both eigenvalues of the structure
tensor using a principal axis transformation.\\
But unlike in \cite{zimmer07}, we aim to to not just keep a small outline around the most important
corners but increase the radius and keep a disc of pixels around each of them. 
\begin{equation}
    \texttt{corner\_region}_R(\boldsymbol x) := \lbrace \boldsymbol y \in \Omega\ : \ \lVert \boldsymbol
    x - \boldsymbol y\rVert_2^2 \leq R^2 \rbrace
\end{equation}
One problem of this approach that was already apparent in \cite{zimmer07} is the sparsity of
corners or similar features in images. Therefore, we had to come up with a solution to somehow
solve this issue in order to create masks with which we can reconstruct the image reasonably 
well.
Another issue is that we have to limit the number of corners depending on the size of the corner
regions since too many corners would result in a very dense mask not very suitable and/or
useful for image compression. 
To solve these issues reasonably well, we introduced two additional steps to the corner detection
process.
\subsection{Circular non-maximum suppression}\label{sub:Suppression}
In the original version of the Foerstner-Harris corner detector, corners are detected as the local
maxima of the cornerness measure \eqref{def:Harris}. However, in this method, the local maxima were
only determined in a 4- or 8-neigbourhood. This works fine in a setting where corner regions are
very small as they were in \cite{zimmer07}. When using larger regions however, this might create a
very unique problem.\\
What we noticed when using the usual 8-neighbourhood non-maximum suppression together with larger
circular corner regions is that in some
images where multiple corners were located right next to each other this approach would create
masks with many overlapping corner regions in a single spot and no regions in other places. To
counteract this, we introduced \textit{circular non-maximum suppression} (\textbf{CNMS}).\\
\begin{figure}
    \begin{lstlisting}[language=Python]
def circular_suppression(harris, (x, y), r, out):
    ''' Circular non-maximum suppression at location (x,y) with radius r.
    Parameters:
        -harris: Harris measure map
        -(x, y): current location
        -r: radius of corner regions
        -out: map of accepted corners after suppression
    '''
    for x', y' in circle(x, y, r):
        if harris[x',y'] > harris[x,y]:
            # The current location is not a maximum
            out[x, y] = 0
            return
    # Current location is a maximum, keep the region around it
    out[x, y] = harris[x, y]
    return 
    \end{lstlisting}
    \caption{Pseudo-Code for circular non-maximum suppression}
\end{figure}
The idea behind this addition is that in larger corner regions, one might have multiple 
important corners in a single corner region.
Now, instead of creating a disc around every single one of these corners, we checked if there 
would be another maximum in an imaginative circle around this corner. 
If this is the case, we disregard this corner region since the current corner is already 
contained in another region.
Using this approach, we were able to create masks with little to no overlap between the
corner regions. However, this introduced one problem that we were not able to fix properly as
discussed in section \ref{sec:Discussion}.
% Examples for circular non-maximum suppression (CNMS) against normal non-maximum suppression
After circular non-maximum suppression, we applied \textit{percentile thresholding} to make sure,
that, out of the set of corners we just detected, we only keep the most important corner regions. 

\begin{figure}
    \centering
    \includegraphics[width=0.31\linewidth]{../Images/trui_corners_cnms.png}
    \includegraphics[width=0.31\linewidth]{../Images/trui-mask_cnms.png}
    \includegraphics[width=0.31\linewidth]{../Images/trui-inpaint_cnms.png}\\
    \vspace{0.2cm}
    \includegraphics[width=0.31\linewidth]{../Images/trui_corners_non_cnms.png}
    \includegraphics[width=0.31\linewidth]{../Images/trui-mask_non_cnms.png}
    \includegraphics[width=0.31\linewidth]{../Images/trui-inpaint_non_cnms.png}
    \caption{Effect of circular non maximum suppression on spread of corners across the image.
        \textbf{Top row:} Position of corners, inpainting mask and inpainting results
        \textbf{with} CNMS.
\textbf{Bottom row:} Position of corners, inpainting mask and inpainting results \textbf{without} CNMS.
    Corner detection with Foerstner-Harris corner detector, $\sigma=1,\rho=2.5,R=15$ and a
percentile of 0.5}
\end{figure}

\TODO{Redo this section for the new thresholding}
\subsection{Percentile thresholding}\label{sub:Percentile}
In the classic version of Harris corner detection, an artificial threshold parameter $T$ is
introduced to weed out `bad' corners. This parameter however is fairly sensible to the input image.
% Images for corner detection with different images and same threshold to prove the point
 As a workaround to make this parameter a bit more robust, we introduced a percentile
parameter that is in turn used to compute a more robust threshold.\\
In statistics, the \textit{n-th percentile} of a set is the value that is larger than $n$ percent
of all values in this set.
We computed the percentile using the \textit{nearest-rank method} since this was the easiest
approach and worked good enough already. In this method, the n-th percentile is just simply
computed as the value at position $\lceil \frac{n}{100}\cdot N_xN_y\rceil$ of the ordered set of values.\\
Since in an image, there are more non-corners than actual corners, we have to deal with many zero
or close-to-zero cornerness values. This is a problem because the percentile computation would be skewed
heavily if e.g. more than 80 percent of all values are zero already. To solve this, we first filtered
out all values below a threshold close to zero and then accumulated the remaining values into an
array of appropriate size. This array is then sorted using an already implemented Quicksort
algorithm from the C standard library. Subsequently, the new threshold is given as the value at
the index calculated above.\\
\begin{figure}[ht]
    \lstinputlisting[language=C, linerange={537-564,566-576}]{/home/daniel/Uni/Thesis/src/corner_detection.c}
    \caption{Percentile thresholding}
\end{figure}

With these two additions we were able to make sure that the resulting masks are not too dense and
the data is reasonably distributed across the image as seen in the examples in \ref{sec:Results}.
Another interesting part is that thanks to CNMS, the number of corners detected is now somewhat
dependent on the size of the corner regions which is a very useful result as it takes care of the
second problem mentioned in the preface to this chapter on the size of the corner regions which is
a very useful result as it takes care of the second problem mentioned in the preface to this
chapter (cf. section \ref{sec:Results}).
% Before/After images
\section{Graphical User Interface (GUI)}
To make experimenting with different parameters and images much simpler, we implemented a Graphical
User Interface using Python's \texttt{Tkinter} module.
We wrote wrapper functions that call the compiled C programme from the command line with the given
parameters.
\begin{figure}
    \includegraphics[width=\linewidth]{../Images/gui.png}
    \caption{GUI for easier testing}
\end{figure}
\begin{figure}
    \includegraphics[width=0.5\linewidth]{../Images/parameters1.png}
    \includegraphics[width=0.5\linewidth]{../Images/parameters2.png}
    \caption{Panels to control parameters}
\end{figure}

\newcommand*{\addheight}[2][.5ex]{%
  \raisebox{0pt}[\dimexpr\height+(#1)\relax]{#2}%
}

\chapter{Experiments}\label{ch:Experiments} 
In this chapter, we will present the experiments conducted to test and verify the methods I
introduced. Furthermore, we will explain how to figure out what the best parameters for both corner detection and inpainting
are. Below, we can see all the test images that were used for the experiments.
\begin{figure}[h]
    \centering
    \fbox{\includegraphics[width=0.2\linewidth]{../../images/binary/rect_tiny.png}}
    \fbox{\includegraphics[width=0.2\linewidth]{../../images/binary/flatcorner.png}}
    \fbox{\includegraphics[width=0.2\linewidth]{../../images/binary/corner.png}}\\\vspace{0.2cm}
    \fbox{\includegraphics[width=0.2\linewidth]{../../images/binary/abstract1_small.png}}
    \fbox{\includegraphics[width=0.2\linewidth]{../../images/binary/cat.png}}
    \fbox{\includegraphics[width=0.2\linewidth]{../../images/binary/ellipse_small.png}}
    \caption{Binary test images. From top left to bottom right: \texttt{rect\_tiny},
    \texttt{flatcorner1}, \texttt{corner}, \texttt{abstract1\_small}, \texttt{cat},
\texttt{ellipse\_small}. Image \texttt{cat} provided by Joachim Weickert. Remaining images created
with GIMP}\label{fig:BinaryImages}
\end{figure}
\begin{figure}[h]
    \centering
    \includegraphics[width=0.3\linewidth]{../../images/grey/house.png}
    \includegraphics[width=0.3\linewidth]{../../images/grey/bank.png}
    \includegraphics[width=0.3\linewidth]{../../images/grey/trui.png}
    \caption{(8-bit) Grayscale test images. From left to right: \texttt{house}, \texttt{bank},
    \texttt{trui}. Images provided by Joachim Weickert.}\label{fig:GreyImages}
\end{figure}
\section{Parameter Selection}\label{sec:ParameterSelection}
Both the corner detection process and the inpainting process introduce many parameters that need to
be chosen carefully to get the best results. The focus in this thesis lies mainly on examining the
choice of parameters for the corner localisation. While we will still explore how the parameters in
the reconstruction step influence the quality of the end result, it will interesting be to
see what role the mask radius plays in this context.

\subsection{Corner Detection}\label{sec:CornerEx}
In the corner detection phase, we differentiate between 4 parameters:
\begin{itemize}
    \item the noise scale $\sigma$
    \item the integration scale $\rho$
    \item the percentile parameter $p$ and
    \item the mask radius (also radius for circular non-maximum suppression) $R$
\end{itemize}
Depending on whether one chooses to use the circular non-maximum suppression (CNMS) and the total pixel
percentage thresholding (TPPT), the mask radius and percentile parameter will have different or additional
meaning.
The percentile parameter determines either the percentage of corners (without TPPT) to keep or the maximum bound
on the pixel density in the inpainting mask (with TPPT).
The mask radius also serves as the radius for the neighbourhood in the CNMS procedure.\\
Regarding the noise scale, we generally want to choose it as large as necessary but keep it as small as
possible, meaning that we want to choose the smallest noise scale that gets rid of most of the
noise in the image since with a larger $\sigma$ 
one often faces the problem that the detected corners can not be located as accurately anymore,
since more and more relevant features are smoothed away (see~\ref{sub:ScaleSpaces}). Another problem
is that the Gaussian scale space (iterated Gaussian smoothing) may even introduce new
corners~\cite{weickert96}.
 Most of the time however, a $\sigma$ of 1 is sufficient enough to remove most of the noise and unnecessary
 details and still provide an accurate result.\\
 The integration scale $\rho$ is an interesting parameter as it basically determines how well the
 corner can be localised. Here, one has to be careful as the choice of this parameter is somewhat
 dependent on the noise scale in the sense that the integration scale always should be 
 at least as large as the noise scale.\\
 On the other hand, by increasing the integration scale, we also increase the inaccuracy in the corner 
 localisation as we see in figure~\ref{fig:Integration}.\\
\begin{figure}[H]
    \centering
    \includegraphics[width=0.4\linewidth]{rect/rect_2_corner.png}
    \includegraphics[width=0.4\linewidth]{rect/rect_4_corner.png}
    \includegraphics[width=0.4\linewidth]{rect/rect_6_corner.png}
    \includegraphics[width=0.4\linewidth]{rect/rect_8_corner.png}
    \caption{Location of corners with different values for $\rho$.\\
\textbf{Top left:} $\rho=2$, \textbf{Top right:} $\rho=4$, \textbf{Bottom left:}
$\rho=6$, \textbf{Bottom right} $\rho=8$. Original image: \texttt{rect\_tiny} (see Figure~\ref{fig:BinaryImages})}\label{fig:Integration}
\end{figure}
\noindent This demonstrates the issue mentioned in Chapter~\ref{ch:RelatedWork} with Zimmer's approach~\cite{zimmer07} quite nicely. 
We can see that with increasing integration scale, the inaccuracy increases
steadily. The problem with Zimmer's approach is that he did not adress these inaccuracies in
their mask generation. Zimmer fixed the mask radius to be just a small neighbourhood of 4 to 8 pixels.
To adjust the amount of corners introduced into the mask, he adapted the integration scale.
Obviously this introduces the aforementioned inaccuracies and combined with the fixed neighbourhood
size, can lead to the actual corner not even being included in the final mask (see Figure
~\ref{fig:Inacc}).
\begin{figure}[h]
    \centering
    \includegraphics[width=0.4\linewidth]{rect/rect_2_mask_position.png}
    \includegraphics[width=0.4\linewidth]{rect/rect_4_mask_position.png}
    \caption{Position of corner regions for examples from Figure~\ref{fig:Integration}.
        Position of the corner in white. Size of the corner region: 9px (8-neighbourhood)
    \textbf{Left:} $\rho=2$, \textbf{Right:} $\rho=4$ (Corner regions in grey for visualisation
purposes, not the actual mask). Original image: \texttt{rect\_tiny} (see Figure~\ref{fig:BinaryImages})}\label{fig:Inacc}
\end{figure}
We will discuss the optimal choice for the mask radius in the next section (\ref{sec:MaskEx}), but
first we will show some examples demonstrating the ideas behind the methods introduced in
Section~\ref{sec:Percentile} and~\ref{sec:Suppression}. As we can see in
figure~\ref{fig:CNMSExample}, CNMS allows us to distribute the corner regions in the mask more
evenly across the whole image. \\\ \\
While in the example without CNMS, most of the corners are
overlapping and grouped up in the lower half of the image; using CNMS, the overlapping corners are
removed and free up slots for corners with a lower cornerness measure, i.e.\ less sharp corners in
the upper half. Later in Figure~\ref{fig:AbstractCNMSExamples}, we can also observe that thanks to
CNMS, a few extra slots become available to be used for other regions.
\begin{figure}
    \centering
    \includegraphics[width=0.31\linewidth]{trui/trui_corners_cnms.png}
    \includegraphics[width=0.31\linewidth]{trui/trui-mask_cnms.png}
    \includegraphics[width=0.31\linewidth]{trui/trui-inpaint_cnms.png}\\
    \vspace{0.2cm}
    \includegraphics[width=0.31\linewidth]{trui/trui_corners_non_cnms.png}
    \includegraphics[width=0.31\linewidth]{trui/trui-mask_non_cnms.png}
    \includegraphics[width=0.31\linewidth]{trui/trui-inpaint_non_cnms.png}
    \caption{Effect of circular non maximum suppression on spread of corners across the image.
        \textbf{Top row:} Position of corners, inpainting mask and inpainting results
        \textbf{with} CNMS\@.
\textbf{Bottom row:} Position of corners, inpainting mask and inpainting results \textbf{without}
CNMS\@.
    Corner detection with Foerstner-Harris corner detector, $\sigma=1,\rho=2.5,R=15$ and a
percentile of 0.5. Original image: \texttt{trui} (see Figure~\ref{fig:GreyImages})}\label{fig:CNMSExample}
\end{figure}
\begin{figure}[h]
    \centering
    \includegraphics[width=0.4\linewidth]{abstract/abstract1_small-mask.png}\hspace{0.2cm}
    \includegraphics[width=0.4\linewidth]{abstract/abstract1_small-inpaint.png}\\
    \vspace*{0.2cm}
    \includegraphics[width=0.4\linewidth]{abstract/abstract1_small-mask_no_cnms.png}\hspace{0.2cm}
    \includegraphics[width=0.4\linewidth]{abstract/abstract1_small-inpaint_no_cnms.png}\\
    \caption{Demonstration of the effects of CNMS on the spread of corner regions.
\textbf{Top row:} $\sigma=1,\rho=1,R=4,q=0.02$, with CNMS, Pixel density: 1.96\%, PSNR\@:
31.45
\textbf{Bottom row:} $\sigma=1,\rho=1,R=4,q=0.02$, without CNMS, Pixel density:
1.51\%, PSNR\@: 21.12
Inpainting parameters: $\sigma=2,\lambda=0.1,\alpha=0.49,\gamma=1,N=1000$
Original image: \texttt{abstract1\_small} (see Figure~\ref{fig:BinaryImages})}\label{fig:AbstractCNMSExamples}
\end{figure}
\begin{figure}[ht]
    \centering
    \includegraphics[width=0.3\linewidth]{tppt_ex/abstract1_small4.png}
    \includegraphics[width=0.3\linewidth]{tppt_ex/abstract1_small8.png}
    \includegraphics[width=0.3\linewidth]{tppt_ex/abstract1_small12.png}
    \caption{Effect of TPPT\@. Upper bound on the pixel density: 2\%. Actual values:
        1.95\%, 1.97\%, 1.96\%. Original image: \texttt{abstract1\_small} (see
    Figure~\ref{fig:BinaryImages})}\label{fig:TPPTEx}
\end{figure}
\\\noindent In Figure~\ref{fig:TPPTEx}, we can also see that TPPT actually produces masks of a
similar size, bound by the percentage given by the parameter $p$. What we also noticed is that the
percentage is not as accurate when using a normal non-maximum suppression approach. This is due to
the potential overlap that is not accounted for in the initial estimation of the amount of corner
regions that can be introduced into the mask.
However, as already mentioned, the TPPT approach is not perfect.
It does not always yield accurate masks since for one
reason, the maximum possible pixel density depends on the amount of corners detected. On the other
hand, with increasing mask radius it is also harder to get an accurate pixel density. Some possible
remedies would include having variable mask radii, inserting random pixels as a post processing
step to fill up the
remaining percentage or, instead of using the parameter as an upper bound, simply minimising the
squared error (i.e.\ allow overshoots).
\newpage\noindent
\subsection{Inpainting}\label{sec:InpaintingEx}
As already mentioned, we are using an EED-based inpainting algorithm with a Charbonnier diffusivity
as explained in Section~\ref{sec:Inpainting}.
The main parameters required by the inpainting process are 
\begin{itemize}
    \item the noise scale $\sigma$,
    \item the contrast parameter $\lambda$, 
    \item the dissipativity parameter $\alpha$ and
    \item a non-negativity parameter $\gamma$.
\end{itemize}
\newpage\noindent
The parameters $\alpha$ and $\gamma$ are purely numerical parameters that are used to 
stabilise the algorithm or rather help to ensure that the stencil weights of the discretisation of
the diffusion process meet certain requirements.
In this work we fixed these parameters to $\alpha=0.49,\, \gamma=1.0$~\cite{conversation}.
In general, one could image the parameter $\alpha$ as a sharpness parameter: the larger the
$\alpha$ (but not larger than 0.5), the sharper the image. More on the nature of these two parameters can be read about in
~\cite{www13, weickert96}.\\
More interesting is the contrast parameter $\lambda$ that is required in the diffusivity function
\eqref{def:Diffusivity}. As already explained in Section~\ref{sec:Structure}, this parameter helps to
distinguish between edges and non-edges. For the EED inpainting this is especially important since
it basically determines how strongly edges will be continued into inpainting regions.
The caveat with this parameter is that even though a smaller $\lambda\ (\leq0.1)$ yields very sharp
results, especially for binary images, it increases the chance of the algorithm becoming unstable.
Normally a $\lambda<0.1$ should not be used~\cite{conversation}. Since the optimal choice for this parameter is still a
topic of current research, we had to experiment a little to find the best choice. Furthermore it was
suggested that the optimal $\lambda$ depends on the input image~\cite{schmaltz14}. 
The noise scale $\sigma$ was always chosen to be somewhere between 1 and 4 as this seemed to give
the best results.
\section{Mask Radius}\label{sec:MaskEx}
By introducing the mask radius as an additional parameter, we now have to pose the question of what
radius would yield the best result. One hypothesis is that it would be best to compute the the
mask radius in dependence of the integration scale $\rho$, in order to account for the inaccuracies
mentioned in Section~\ref{sec:CornerEx}~\cite{conversation}.
\begin{figure}[h]
    \centering
    \includegraphics[width=0.2\linewidth]{disctest/flatcorner1.png}\hspace{0.2cm}
    \includegraphics[width=0.2\linewidth]{disctest/flatcorner1mask.png}\hspace{0.2cm}
    \includegraphics[width=0.2\linewidth]{disctest/flatcorner1inpaint.png}\\
    \vspace*{0.2cm}
    \includegraphics[width=0.2\linewidth]{disctest/flatcorner2.png}\hspace{0.2cm}
    \includegraphics[width=0.2\linewidth]{disctest/flatcorner2mask.png}\hspace{0.2cm}
    \includegraphics[width=0.2\linewidth]{disctest/flatcorner2inpaint.png}\\
    \vspace*{0.2cm}
    \includegraphics[width=0.2\linewidth]{disctest/flatcorner3.png}\hspace{0.2cm}
    \includegraphics[width=0.2\linewidth]{disctest/flatcorner3mask.png}\hspace{0.2cm}
    \includegraphics[width=0.2\linewidth]{disctest/flatcorner3inpaint.png}\\
    \vspace*{0.2cm}
    \includegraphics[width=0.2\linewidth]{disctest/flatcorner4.png}\hspace{0.2cm}
    \includegraphics[width=0.2\linewidth]{disctest/flatcorner4mask.png}\hspace{0.2cm}
    \includegraphics[width=0.2\linewidth]{disctest/flatcorner4inpaint.png}\\
    \vspace*{0.2cm}
    \includegraphics[width=0.2\linewidth]{disctest/flatcorner5.png}\hspace{0.2cm}
    \includegraphics[width=0.2\linewidth]{disctest/flatcorner5mask.png}\hspace{0.2cm}
    \includegraphics[width=0.2\linewidth]{disctest/flatcorner5inpaint.png}\\
    \caption{Experiment with identical mask radius $R=4$ and varying integration scale. $\rho$ from top to bottom:
    1, 2, 3, 4, 5. PSNR from top to bottom: 12.49, 12.49, 12.49, 11.31, 8.09. Original image:
\texttt{flatcorner} (see Figure~\ref{fig:BinaryImages})}\label{fig:MaskEx}
\end{figure}
\newpage\noindent The hypothesis was tested by replacing the Gaussian
convolution in the computation of the structure tensor in the corner detection process for a
regular convolution with a so called \textit{normalised pillbox kernel}, which is basically just
averaging inside a disk-shaped neighbourhood, to get a better representation of the uncertainty in
the corner localisation~\cite{conversation}. After that, it was tested whether it would be
sensible in terms of reconstruction quality to choose a mask radius equal to the radius of the
pillbox kernel~\cite{conversation}. 
In Figure~\ref{fig:MaskEx} we can see the motivation behind the
idea. There we can see how the actual corner slowly moves out of the corner region for increasing
$\rho$. We should note that the corner locations and thus the masks and inpainting results for
$\rho=1,2,3$ are the same. The interesting thing however is that the quality suddenly drops when we 
have 
\begin{equation}
    \frac{R}{\rho}\leq1
\end{equation}
This observation motivated the next experiment in which we tested different integration scales
paired with multiple mask radii to create the matrix we can see in Figure~\ref{fig:InaccTestMask}.
Looking at the resulting inpainted images (see Figure~\ref{fig:InaccTestInpaint}) and their
respective PSNR values (see Figure~\ref{fig:InaccTestPSNR}), we observe a similar effect, namely
that the PSNR jumps by a large margin every time the ratio between the mask radius and the
integration scale goes beyond 1.
Another interesting observation is that the biggest jump is almost
always at a transition from a ratio smaller than 1 to a ratio of 1 (cells in grey), suggesting that
as soon as the inaccuracy is matched by the mask radius, the inpainting quality increases. The only
exception here is $\rho=3$. A possible explanation is that in this case, the corner
location is the same for both $\rho=2$ and $\rho=3$, which is why all the inpainted images and
masks are identical. Similar results can be observed in
Figure~\ref{fig:RectInaccTestPSNR}.\\
All in all we can say that the experiments provided reasonable proof that the hypothesis stated
above is true. As a corollary from this we can deduce that a more accurate corner detection method
could result in smaller mask sizes and ultimately better compression rates.
\begin{figure}[H]
    \centering
    \begin{tabular}{|c|cccccc|}
        \hline
        \diagbox{$\rho$}{$R$}&1&2&3&4&5&6\\\hline
        \raisebox{0.8cm}{1} &
        \addheight{\includegraphics[width=0.12\linewidth]{inacctest/cornerr1m1mask.png}}
          &
        \addheight{\includegraphics[width=0.12\linewidth]{inacctest/cornerr1m2mask.png}}
          &
        \addheight{\includegraphics[width=0.12\linewidth]{inacctest/cornerr1m3mask.png}}
          &
        \addheight{\includegraphics[width=0.12\linewidth]{inacctest/cornerr1m4mask.png}}
          &
        \addheight{\includegraphics[width=0.12\linewidth]{inacctest/cornerr1m5mask.png}}
          &
        \addheight{\includegraphics[width=0.12\linewidth]{inacctest/cornerr1m6mask.png}}\\
        \raisebox{0.8cm}{2} &
        \addheight{\includegraphics[width=0.12\linewidth]{inacctest/cornerr2m1mask.png}}
          &
        \addheight{\includegraphics[width=0.12\linewidth]{inacctest/cornerr2m2mask.png}}
          &
        \addheight{\includegraphics[width=0.12\linewidth]{inacctest/cornerr2m3mask.png}}
          &
        \addheight{\includegraphics[width=0.12\linewidth]{inacctest/cornerr2m4mask.png}}
          &
        \addheight{\includegraphics[width=0.12\linewidth]{inacctest/cornerr2m5mask.png}}
          &
        \addheight{\includegraphics[width=0.12\linewidth]{inacctest/cornerr2m6mask.png}}\\
        \raisebox{0.8cm}{3} &
        \addheight{\includegraphics[width=0.12\linewidth]{inacctest/cornerr3m1mask.png}}
          &
        \addheight{\includegraphics[width=0.12\linewidth]{inacctest/cornerr3m2mask.png}}
          &
        \addheight{\includegraphics[width=0.12\linewidth]{inacctest/cornerr3m3mask.png}}
          &
        \addheight{\includegraphics[width=0.12\linewidth]{inacctest/cornerr3m4mask.png}}
          &
        \addheight{\includegraphics[width=0.12\linewidth]{inacctest/cornerr3m5mask.png}}
          &
        \addheight{\includegraphics[width=0.12\linewidth]{inacctest/cornerr3m6mask.png}}\\
        \raisebox{0.8cm}{4} &
        \addheight{\includegraphics[width=0.12\linewidth]{inacctest/cornerr4m1mask.png}}
          &
        \addheight{\includegraphics[width=0.12\linewidth]{inacctest/cornerr4m2mask.png}}
          &
        \addheight{\includegraphics[width=0.12\linewidth]{inacctest/cornerr4m3mask.png}}
          &
        \addheight{\includegraphics[width=0.12\linewidth]{inacctest/cornerr4m4mask.png}}
          &
        \addheight{\includegraphics[width=0.12\linewidth]{inacctest/cornerr4m5mask.png}}
          &
        \addheight{\includegraphics[width=0.12\linewidth]{inacctest/cornerr4m6mask.png}}\\
        \raisebox{0.8cm}{5} &
        \addheight{\includegraphics[width=0.12\linewidth]{inacctest/cornerr5m1mask.png}}
          &
        \addheight{\includegraphics[width=0.12\linewidth]{inacctest/cornerr5m2mask.png}}
          &
        \addheight{\includegraphics[width=0.12\linewidth]{inacctest/cornerr5m3mask.png}}
          &
        \addheight{\includegraphics[width=0.12\linewidth]{inacctest/cornerr5m4mask.png}}
          &
        \addheight{\includegraphics[width=0.12\linewidth]{inacctest/cornerr5m5mask.png}}
          &
        \addheight{\includegraphics[width=0.12\linewidth]{inacctest/cornerr5m6mask.png}}\\\hline
    \end{tabular}
    \caption{Inpainting masks for image corner.pgm for varying integration scale and mask radius.
    Original image: \texttt{corner} (see Figure~\ref{fig:BinaryImages})}\label{fig:InaccTestMask}
\end{figure}
\begin{figure}[H]
    \centering
    \begin{tabular}{|c|cccccc|}
        \hline
        \diagbox{$\rho$}{$R$}&1&2&3&4&5&6\\\hline
        \raisebox{0.8cm}{1} &
        \addheight{\includegraphics[width=0.12\linewidth]{inacctest/cornerr1m1inpaint.png}}
          &
        \addheight{\includegraphics[width=0.12\linewidth]{inacctest/cornerr1m2inpaint.png}}
          &
        \addheight{\includegraphics[width=0.12\linewidth]{inacctest/cornerr1m3inpaint.png}}
          &
        \addheight{\includegraphics[width=0.12\linewidth]{inacctest/cornerr1m4inpaint.png}}
          &
        \addheight{\includegraphics[width=0.12\linewidth]{inacctest/cornerr1m5inpaint.png}}
          &
        \addheight{\includegraphics[width=0.12\linewidth]{inacctest/cornerr1m6inpaint.png}}\\
        \raisebox{0.8cm}{2} &
        \addheight{\includegraphics[width=0.12\linewidth]{inacctest/cornerr2m1inpaint.png}}
          &
        \addheight{\includegraphics[width=0.12\linewidth]{inacctest/cornerr2m2inpaint.png}}
          &
        \addheight{\includegraphics[width=0.12\linewidth]{inacctest/cornerr2m3inpaint.png}}
          &
        \addheight{\includegraphics[width=0.12\linewidth]{inacctest/cornerr2m4inpaint.png}}
          &
        \addheight{\includegraphics[width=0.12\linewidth]{inacctest/cornerr2m5inpaint.png}}
          &
        \addheight{\includegraphics[width=0.12\linewidth]{inacctest/cornerr2m6inpaint.png}}\\
        \raisebox{0.8cm}{3} &
        \addheight{\includegraphics[width=0.12\linewidth]{inacctest/cornerr3m1inpaint.png}}
          &
        \addheight{\includegraphics[width=0.12\linewidth]{inacctest/cornerr3m2inpaint.png}}
          &
        \addheight{\includegraphics[width=0.12\linewidth]{inacctest/cornerr3m3inpaint.png}}
          &
        \addheight{\includegraphics[width=0.12\linewidth]{inacctest/cornerr3m4inpaint.png}}
          &
        \addheight{\includegraphics[width=0.12\linewidth]{inacctest/cornerr3m5inpaint.png}}
          &
        \addheight{\includegraphics[width=0.12\linewidth]{inacctest/cornerr3m6inpaint.png}}\\
        \raisebox{0.8cm}{4} &
        \addheight{\includegraphics[width=0.12\linewidth]{inacctest/cornerr4m1inpaint.png}}
          &
        \addheight{\includegraphics[width=0.12\linewidth]{inacctest/cornerr4m2inpaint.png}}
          &
        \addheight{\includegraphics[width=0.12\linewidth]{inacctest/cornerr4m3inpaint.png}}
          &
        \addheight{\includegraphics[width=0.12\linewidth]{inacctest/cornerr4m4inpaint.png}}
          &
        \addheight{\includegraphics[width=0.12\linewidth]{inacctest/cornerr4m5inpaint.png}}
          &
        \addheight{\includegraphics[width=0.12\linewidth]{inacctest/cornerr4m6inpaint.png}}\\
        \raisebox{0.8cm}{5} &
        \addheight{\includegraphics[width=0.12\linewidth]{inacctest/cornerr5m1inpaint.png}}
          &
        \addheight{\includegraphics[width=0.12\linewidth]{inacctest/cornerr5m2inpaint.png}}
          &
        \addheight{\includegraphics[width=0.12\linewidth]{inacctest/cornerr5m3inpaint.png}}
          &
        \addheight{\includegraphics[width=0.12\linewidth]{inacctest/cornerr5m4inpaint.png}}
          &
        \addheight{\includegraphics[width=0.12\linewidth]{inacctest/cornerr5m5inpaint.png}}
          &
        \addheight{\includegraphics[width=0.12\linewidth]{inacctest/cornerr5m6inpaint.png}}\\\hline
    \end{tabular}
    \caption{Inpainting results for mask shown in Figure~\ref{fig:InaccTestMask}. Inpainting
    parameters: $\sigma=2,\lambda=0.4,\alpha=0.49,\gamma=1$. Original image: \texttt{corner} (see Figure~\ref{fig:BinaryImages})}\label{fig:InaccTestInpaint}
\end{figure}
\begin{figure}[H]
    \centering
    \begin{tabular}{|c|c|c|c|c|c|c|}
        \hline
        \diagbox{$\rho$}{$R$}&1&2&3&4&5&6\\\hline
        1 & \cellcolor{gray!25}\textbf{4.83} & \textbf{10.97} & 15.70 & 20.39 & 22.71 & 23.58 \\\hline
        2 & 1.25 & \cellcolor{gray!25}7.89 & \textbf{17.31} & \textbf{21.48} & \textbf{23.45} &
        24.13 \\\hline
        3 & 1.25 & 7.89 & \cellcolor{gray!25}\textbf{17.31} & \textbf{21.48} & \textbf{23.45} &
        24.13 \\\hline
        4 & 1.25 & 1.25 & 7.21 & \cellcolor{gray!25}16.91 & 22.92 & \textbf{24.34} \\\hline
        5 & 1.25 & 1.25 & 1.25 & 7.84 & \cellcolor{gray!25}17.96 & 22.82 \\\hline
    \end{tabular}
    \caption{PSNR values of the reconstructed images in
    Figure~\ref{fig:InaccTestInpaint}. Bold values are the best in the respective column.
Gray cells are cells where the $R$ and $\rho$ are equal.}\label{fig:InaccTestPSNR}
\end{figure}
\begin{figure}[H]
    \centering
    \begin{tabular}{|c|cccccc|}
        \hline
        \diagbox{$\rho$}{$R$}&1&2&3&4&5&6\\\hline
        \raisebox{0.8cm}{1} &
        \addheight{\includegraphics[width=0.12\linewidth]{rect_inacc/rect_tinyr1m1mask.png}}
          &
        \addheight{\includegraphics[width=0.12\linewidth]{rect_inacc/rect_tinyr1m2mask.png}}
          &
        \addheight{\includegraphics[width=0.12\linewidth]{rect_inacc/rect_tinyr1m3mask.png}}
          &
        \addheight{\includegraphics[width=0.12\linewidth]{rect_inacc/rect_tinyr1m4mask.png}}
          &
        \addheight{\includegraphics[width=0.12\linewidth]{rect_inacc/rect_tinyr1m5mask.png}}
          &
        \addheight{\includegraphics[width=0.12\linewidth]{rect_inacc/rect_tinyr1m6mask.png}}\\
        \raisebox{0.8cm}{2} &
        \addheight{\includegraphics[width=0.12\linewidth]{rect_inacc/rect_tinyr2m1mask.png}}
          &
        \addheight{\includegraphics[width=0.12\linewidth]{rect_inacc/rect_tinyr2m2mask.png}}
          &
        \addheight{\includegraphics[width=0.12\linewidth]{rect_inacc/rect_tinyr2m3mask.png}}
          &
        \addheight{\includegraphics[width=0.12\linewidth]{rect_inacc/rect_tinyr2m4mask.png}}
          &
        \addheight{\includegraphics[width=0.12\linewidth]{rect_inacc/rect_tinyr2m5mask.png}}
          &
        \addheight{\includegraphics[width=0.12\linewidth]{rect_inacc/rect_tinyr2m6mask.png}}\\
        \raisebox{0.8cm}{3} &
        \addheight{\includegraphics[width=0.12\linewidth]{rect_inacc/rect_tinyr3m1mask.png}}
          &
        \addheight{\includegraphics[width=0.12\linewidth]{rect_inacc/rect_tinyr3m2mask.png}}
          &
        \addheight{\includegraphics[width=0.12\linewidth]{rect_inacc/rect_tinyr3m3mask.png}}
          &
        \addheight{\includegraphics[width=0.12\linewidth]{rect_inacc/rect_tinyr3m4mask.png}}
          &
        \addheight{\includegraphics[width=0.12\linewidth]{rect_inacc/rect_tinyr3m5mask.png}}
          &
        \addheight{\includegraphics[width=0.12\linewidth]{rect_inacc/rect_tinyr3m6mask.png}}\\
        \raisebox{0.8cm}{4} &
        \addheight{\includegraphics[width=0.12\linewidth]{rect_inacc/rect_tinyr4m1mask.png}}
          &
        \addheight{\includegraphics[width=0.12\linewidth]{rect_inacc/rect_tinyr4m2mask.png}}
          &
        \addheight{\includegraphics[width=0.12\linewidth]{rect_inacc/rect_tinyr4m3mask.png}}
          &
        \addheight{\includegraphics[width=0.12\linewidth]{rect_inacc/rect_tinyr4m4mask.png}}
          &
        \addheight{\includegraphics[width=0.12\linewidth]{rect_inacc/rect_tinyr4m5mask.png}}
          &
        \addheight{\includegraphics[width=0.12\linewidth]{rect_inacc/rect_tinyr4m6mask.png}}\\
        \raisebox{0.8cm}{5} &
        \addheight{\includegraphics[width=0.12\linewidth]{rect_inacc/rect_tinyr5m1mask.png}}
          &
        \addheight{\includegraphics[width=0.12\linewidth]{rect_inacc/rect_tinyr5m2mask.png}}
          &
        \addheight{\includegraphics[width=0.12\linewidth]{rect_inacc/rect_tinyr5m3mask.png}}
          &
        \addheight{\includegraphics[width=0.12\linewidth]{rect_inacc/rect_tinyr5m4mask.png}}
          &
        \addheight{\includegraphics[width=0.12\linewidth]{rect_inacc/rect_tinyr5m5mask.png}}
          &
        \addheight{\includegraphics[width=0.12\linewidth]{rect_inacc/rect_tinyr5m6mask.png}}\\\hline
    \end{tabular}
    \caption{Inpainting masks for for varying integration scale and mask radius. Original image:
    \texttt{rect\_tiny} (see Figure~\ref{fig:BinaryImages})}\label{fig:RectInaccTestMask}
\end{figure}
\begin{figure}[H]
    \centering
    \begin{tabular}{|c|cccccc|}
        \hline
        \diagbox{$\rho$}{$R$}&1&2&3&4&5&6\\\hline
        \raisebox{0.8cm}{1} &
        \addheight{\includegraphics[width=0.12\linewidth]{rect_inacc/rect_tinyr1m1inpaint.png}}
          &
        \addheight{\includegraphics[width=0.12\linewidth]{rect_inacc/rect_tinyr1m2inpaint.png}}
          &
        \addheight{\includegraphics[width=0.12\linewidth]{rect_inacc/rect_tinyr1m3inpaint.png}}
          &
        \addheight{\includegraphics[width=0.12\linewidth]{rect_inacc/rect_tinyr1m4inpaint.png}}
          &
        \addheight{\includegraphics[width=0.12\linewidth]{rect_inacc/rect_tinyr1m5inpaint.png}}
          &
        \addheight{\includegraphics[width=0.12\linewidth]{rect_inacc/rect_tinyr1m6inpaint.png}}\\
        \raisebox{0.8cm}{2} &
        \addheight{\includegraphics[width=0.12\linewidth]{rect_inacc/rect_tinyr2m1inpaint.png}}
          &
        \addheight{\includegraphics[width=0.12\linewidth]{rect_inacc/rect_tinyr2m2inpaint.png}}
          &
        \addheight{\includegraphics[width=0.12\linewidth]{rect_inacc/rect_tinyr2m3inpaint.png}}
          &
        \addheight{\includegraphics[width=0.12\linewidth]{rect_inacc/rect_tinyr2m4inpaint.png}}
          &
        \addheight{\includegraphics[width=0.12\linewidth]{rect_inacc/rect_tinyr2m5inpaint.png}}
          &
        \addheight{\includegraphics[width=0.12\linewidth]{rect_inacc/rect_tinyr2m6inpaint.png}}\\
        \raisebox{0.8cm}{3} &
        \addheight{\includegraphics[width=0.12\linewidth]{rect_inacc/rect_tinyr3m1inpaint.png}}
          &
        \addheight{\includegraphics[width=0.12\linewidth]{rect_inacc/rect_tinyr3m2inpaint.png}}
          &
        \addheight{\includegraphics[width=0.12\linewidth]{rect_inacc/rect_tinyr3m3inpaint.png}}
          &
        \addheight{\includegraphics[width=0.12\linewidth]{rect_inacc/rect_tinyr3m4inpaint.png}}
          &
        \addheight{\includegraphics[width=0.12\linewidth]{rect_inacc/rect_tinyr3m5inpaint.png}}
          &
        \addheight{\includegraphics[width=0.12\linewidth]{rect_inacc/rect_tinyr3m6inpaint.png}}\\
        \raisebox{0.8cm}{4} &
        \addheight{\includegraphics[width=0.12\linewidth]{rect_inacc/rect_tinyr4m1inpaint.png}}
          &
        \addheight{\includegraphics[width=0.12\linewidth]{rect_inacc/rect_tinyr4m2inpaint.png}}
          &
        \addheight{\includegraphics[width=0.12\linewidth]{rect_inacc/rect_tinyr4m3inpaint.png}}
          &
        \addheight{\includegraphics[width=0.12\linewidth]{rect_inacc/rect_tinyr4m4inpaint.png}}
          &
        \addheight{\includegraphics[width=0.12\linewidth]{rect_inacc/rect_tinyr4m5inpaint.png}}
          &
        \addheight{\includegraphics[width=0.12\linewidth]{rect_inacc/rect_tinyr4m6inpaint.png}}\\
        \raisebox{0.8cm}{5} &
        \addheight{\includegraphics[width=0.12\linewidth]{rect_inacc/rect_tinyr5m1inpaint.png}}
          &
        \addheight{\includegraphics[width=0.12\linewidth]{rect_inacc/rect_tinyr5m2inpaint.png}}
          &
        \addheight{\includegraphics[width=0.12\linewidth]{rect_inacc/rect_tinyr5m3inpaint.png}}
          &
        \addheight{\includegraphics[width=0.12\linewidth]{rect_inacc/rect_tinyr5m4inpaint.png}}
          &
        \addheight{\includegraphics[width=0.12\linewidth]{rect_inacc/rect_tinyr5m5inpaint.png}}
          &
        \addheight{\includegraphics[width=0.12\linewidth]{rect_inacc/rect_tinyr5m6inpaint.png}}\\\hline
    \end{tabular}
    \caption{Inpainting results for mask shown in Figure~\ref{fig:RectInaccTestMask}. Inpainting
    parameters: $\sigma=2,\lambda=0.4,\alpha=0.49,\gamma=1$. Original image:
    \texttt{rect\_tiny} (see Figure~\ref{fig:BinaryImages})}\label{fig:RectInaccTestInpaint}
\end{figure}
\begin{figure}[H]
    \centering
    \begin{tabular}{|c|c|c|c|c|c|c|}
            \hline
            \diagbox{$\rho$}{$R$}&1&2&3&4&5&6\\\hline
            1 & \cellcolor{gray!25}\textbf{4.76} & 6.26 & 7.22 & 19.83 & 23.08 & 24.64 \\\hline
            2 & 1.93 & \cellcolor{gray!25}\textbf{9.71} & \textbf{19.41} & \textbf{22.86} & 24.54 & 25.63 \\\hline
            3 & 1.93 & \textbf{9.71} & \cellcolor{gray!25}\textbf{19.41} & \textbf{22.86} & 24.54 & 25.63 \\\hline
            4 & 1.93 & 1.93 & 9.51 & \cellcolor{gray!25}21.29 & \textbf{25.35} & 26.59 \\\hline
            5 & 1.93 & 1.93 & 1.93 & 14.39 & \cellcolor{gray!25}24.03 & \textbf{26.81} \\\hline
    \end{tabular}
    \caption{PSNR values of the reconstructed images in
        Figure~\ref{fig:RectInaccTestInpaint}. Bold values are the best in the respective column.
Gray cells are cells where the $R$ and $\rho$ are equal.}\label{fig:RectInaccTestPSNR}
\end{figure}
\newpage
\section{Examples}\label{sec:Results}
Finally, we will present some results of the reconstruction process using only corner regions as
seed points.
First up, we want to demonstrate the approach with multiple binary images shown in
Figure~\ref{fig:BinaryImages}. As we can see in Figure~\ref{fig:AbstractInpainting}, Figure~\ref{fig:RectInaccTestInpaint} and Figure~\ref{fig:InaccTestInpaint}, the algorithm works fairly well especially for binary
images with clearly defined corners even though the edges are not as sharp as one might have
hoped. However, the inpainting parameters were not highly optimised in these examples. The purpose
of these experiments was to show the influence of the ratio between mask radius and integration
scale on the final result. As a more practical example we see in Figure~\ref{fig:CatExample} that
piecewise straight edges can be reconstructed fairly well, whereas longer curves can be a challenge
for the algorithm.  On the other hand, curves pose a problem: the algorithm can
not really detect any corners there because of the small curvature (see Figure~\ref{fig:CatExample}). 
Thus, we have too little information to reconstruct this area properly. 
Finally, we can see that the method still works well, even with a bit of 
noise applied to it (see Figure~\ref{fig:AbstractNoise1})
However, if one increases the noise level we observe that the quality of the reconstruction
deteriorates further, which is to be expected.
\begin{figure}[ht]
    \centering
    \vspace*{0.2cm}
    \includegraphics[width=0.4\linewidth]{abstract/abstract1_small-mask_larger_radius.png}\hspace{0.2cm}
    \includegraphics[width=0.4\linewidth]{abstract/abstract1_small-inpaint_larger_radius.png}
    \caption{Inpainting of image \texttt{abstract1\_small}. $\sigma=1,\rho=1,R=7,q=0.02$, Pixel
        density: 1.99\%, PSNR\@: 24.57, Inpainting parameters:
        $\sigma=2,\lambda=0.1,\alpha=0.49,\gamma=1,N=1000$, Original image:
        \texttt{abstract1\_small} (see Figure~\ref{fig:BinaryImages})}\label{fig:AbstractInpainting}
\end{figure}
\begin{figure}[ht]
    \centering
    \includegraphics[width=0.29\linewidth]{../../images/binary/abstract1_less_noise.png}
    \includegraphics[width=0.29\linewidth]{abstract/abstract1_less_noise-mask.png}
    \includegraphics[width=0.29\linewidth]{abstract/abstract1_less_noise-inpaint.png}
    \caption{Example of inpainting with noise degradation. Corner detection parameters:
    $\sigma=1.5,\rho=2.0,R=6,q=0.02$, Pixel density: 1.88\%, Inpainting parameters:
    $\sigma=2,\lambda=0.2,\alpha=0.49,\gamma=1$, PSNR\@: 20.74. Original image (without noise):
        \texttt{abstract1\_small} (see Figure~\ref{fig:BinaryImages})}\label{fig:AbstractNoise1}
\end{figure}
\begin{figure}[ht]
    \centering
    \includegraphics[width=0.29\linewidth]{../../images/binary/abstract1_noise.png}
    \includegraphics[width=0.29\linewidth]{abstract/abstract1_noise-mask.png}
    \includegraphics[width=0.29\linewidth]{abstract/abstract1_noise-inpaint.png}
    \caption{Example of inpainting with noise degradation. Corner detection parameters:
    $\sigma=1.5,\rho=2.0,R=6,q=0.02$, Pixel density: 1.88\%, Inpainting parameters:
$\sigma=2,\lambda=0.2,\alpha=0.49,\gamma=1$, PSNR\@: 16.25. Original image (without noise):
        \texttt{abstract1\_small} (see Figure~\ref{fig:BinaryImages})}\label{fig:AbstractNoise2}
\end{figure}                                       
\\ \noindent If we now look to more textured 8-bit grey value images like \texttt{house} and
\texttt{bank} (see Figure~\ref{fig:GreyImages}), we
see that the inpainting results become a bit worse compared to the binary images. This is primarily
due to the lacking availability of corners in natural images.\\\ \\
In Figure~\ref{fig:HouseEx}, we see that reconstruction using only the corners one would naturally
classify as corners (bottom row) is worse than the reconstruction where we added some corners that
would not qualify as a corner directly. For example the addition of the single corner region in the
top right corner improves the reconstruction in this area by a large margin, despite being more of
a part of an edge.
This same figure is also a prime example for the removal of textures if the textured regions are
underrepresented.
\begin{figure}[htpb]
    \centering
    \includegraphics[width=0.4\linewidth]{bank/bankr2m4mask.png}
    \includegraphics[width=0.4\linewidth]{bank/bankr2m4inpaint.png}\\\vspace{0.2cm}
    \caption{Inpaining of \texttt{bank}. Corner detection parameters:
        $\sigma=1,\rho=2,R=4,q=0.11$,TPPT+CNMS,
    Pixel density: 10.76\%, Inpainting parameters: $\sigma=2,\lambda=0.5,\alpha=0.49,\gamma=1$\\
PSNR:\@18.37. Original image: \texttt{bank} (see Figure~\ref{fig:GreyImages})}\label{fig:BankEx}
\end{figure}
\begin{figure}[h]
    \centering
    \includegraphics[width=0.4\linewidth]{cat/cat-mask1510.png}\hspace{0.2cm}
    \includegraphics[width=0.4\linewidth]{cat/cat-inpaint1510.png}
    \caption{Mask and inpainted result for cat.pgm. Corner detection parameters:
    $\sigma=1,\rho=1,R=10$, pixel density: 4.74\%, Inpainting parameters: $\sigma=2,\lambda=0.2$,
PSNR:\@18.35. Original image: \texttt{cat} (see Figure~\ref{fig:BinaryImages})}\label{fig:CatExample}
\end{figure}
\begin{figure}[h]
    \centering
    \includegraphics[width=0.2\linewidth]{ellipse_small-mask.png}
    \includegraphics[width=0.2\linewidth]{ellipse_small-inpaint.png}
    \caption{Example for the image ellipse\_small (see Figure~\ref{fig:BinaryImages}). Corner
    detection parameters: $\sigma=1,\rho=,R=4$. Inpainting parameters:
$\sigma=2,\lambda=0.1,\alpha=0.49,\gamma=1$. PSNR\@: 26.64. Original image: \texttt{ellipse} (see
Figure~\ref{fig:BinaryImages})}\label{fig:EllipseSmallExample}
\end{figure}
\begin{figure}[ht]
    \centering
    \includegraphics[width=0.4\linewidth]{house/house-mask.png}
    \includegraphics[width=0.4\linewidth]{house/house-inpaint.png}\\\vspace*{0.2cm}
    \includegraphics[width=0.4\linewidth]{house/house-mask3.png}
    \includegraphics[width=0.4\linewidth]{house/house-inpaint3.png}
    \caption{Mask and inpainting results of the image \texttt{house}. 
        \textbf{Top row: }Corner detection parameters: $\sigma=1,\rho=1,R=5,q=0.1, \text{CNMS}$, Pixel density: 9.23\%, Inpainting parameters:
    $\sigma=3,\lambda=0.2,\alpha=0.49,\gamma=1$, PSNR\@: 21.11\\
    \textbf{Bottom row: }Corner detection parameters: $\sigma=1,\rho=1.5,R=5,q=0.1, \text{no
    CNMS}$, Pixel density: 7.71\%, Inpainting parameters:
$\sigma=2,\lambda=0.4,\alpha=0.49,\gamma=1$, PSNR\@: 18.68. Original image: \texttt{house} (see
Figure~\ref{fig:GreyImages})}\label{fig:HouseEx}
\end{figure}

\chapter{Conclusion and Outlook}

\label{ch:Conclusion}

\section{Discussion}
\subsection{What works well\dots}
\subsection{\dots and what does not}
\section{Future Work}
\subsection{Improvements that can be done}
\subsection{Implications for image compression}


%----------------------------------------------------------------------------------------
%   THESIS CONTENT - APPENDICES
%----------------------------------------------------------------------------------------

\appendix % Cue to tell LaTeX that the following "chapters" are Appendices

%----------------------------------------------------------------------------------------
%   BIBLIOGRAPHY
%----------------------------------------------------------------------------------------

\printbibliography

%----------------------------------------------------------------------------------------

\end{document}
